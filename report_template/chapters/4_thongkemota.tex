\section{Thống kê mô tả}

Sau khi đã loại bỏ các các biến có tỉ lệ dữ liệu khuyết cao, nhóm tiến hành chọn lọc các biến quan trọng. Sau quá trình tham khảo, nhóm nhận thấy thông số quyết định hiệu suất xử lí của GPU là băng thông bộ nhớ (Memory\_Bandwidth) vì nó quyết định tốc độ truyền tải dữ liệu giữa bộ nhớ và GPU, đặc biệt với các tác vụ yêu cầu GPU xử lí lượng dữ liệu lớn. Băng thông càng cao thì tốc độ xử lý và khả năng đa nhiệm của hệ thống càng nhanh và ổn định. Ngoài ra, nhóm lựa chọn các biến có tầm ảnh hưởng khác nhau đến Memory\_Bandwidth gồm: Manufacturer, Process, Memory\_Speed, Memory\_Bus, Memory\_Type, L2\_Cache, Dedicated.

\subsection{Tính các giá trị đặc trưng của các biến.}

\subsubsection{Đối với các biến có giá trị số.}
Ta tiến hành lấy các biến trong dữ liệu mẫu ra để tính các giá trị đặc trưng của các biến như sau:

\begin{lstlisting}[language=R, caption=Tính các giá trị đặc trưng của các biến]
numeric_data <- main_df[, c("Memory_Bandwidth", "Process", "Memory_Speed", "Memory_Bus", "L2_Cache")]
sds <- sapply(numeric_data, sd, na.rm = TRUE)
summary(numeric_data)
sds
\end{lstlisting}
\vspace{-20pt}

\underline{\textbf{Kết quả:}}

\begin{center}
    \begin{tcolorbox}[fontupper=\footnotesize]
        \begin{verbatim}
> summary(numeric_data)
Memory_Bandwidth    Process        Memory_Speed    Memory_Bus        L2_Cache     
Min.   :   4.0   Min.   : 14.00   Min.   : 275   Min.   :  32.0   Min.   :   0.0  
1st Qu.:  64.0   1st Qu.: 28.00   1st Qu.: 950   1st Qu.: 128.0   1st Qu.: 256.0  
Median : 134.4   Median : 28.00   Median :1250   Median : 192.0   Median : 512.0  
Mean   : 162.7   Mean   : 32.21   Mean   :1275   Mean   : 222.4   Mean   : 847.4  
3rd Qu.: 224.4   3rd Qu.: 40.00   3rd Qu.:1527   3rd Qu.: 256.0   3rd Qu.:1024.0  
Max.   :1280.0   Max.   :150.00   Max.   :2127   Max.   :4096.0   Max.   :6144.0  
> sds
Memory_Bandwidth    Process        Memory_Speed    Memory_Bus        L2_Cache 
       137.13660   13.76039           395.62258     236.03473       941.24799 
        \end{verbatim}
    \end{tcolorbox}
\end{center}

% \begin{figure}[H]
%     \centering
%     \includegraphics[width=0.8\textwidth]{graphics/4_1tkmt.png}
%     \caption{Kết quả tính các giá trị đặc trưng của các biến}
%     \label{fig:4_1tkmt}
% \end{figure}

\underline{\textbf{Nhận xét:}} từ kết quả ở trên, ta có thể xem giá trị lớn nhất, nhỏ nhất, độ lệch chuẩn và tứ phân vị của các biến cũng như giá trị trung bình của chúng.

Nhóm cũng thử vẽ biểu đồ hộp để trực quan hóa trực quan hoá các số liệu :
\begin{lstlisting}[language=R, caption=Vẽ biểu đồ hộp cho các biến số]
par(mfrow = c(2, 2)) # Chia khung thanh 2 hang va 2 cot

boxplot(main_df$Process, main = "Process (nm)", col = "lightblue", ylab = "Value")
boxplot(main_df$Memory_Speed, main = "Memory Speed (MHz)", col = "lightgreen", ylab = "Value")
boxplot(main_df$Memory_Bus, main = "Memory Bus (Bit)", col = "pink", ylab = "Value")
boxplot(main_df$L2_Cache, main = "L2 Cache (KB)", col = "wheat", ylab = "Value")

par(mfrow = c(1, 1)) # Chinh lai nhu ban dau

\end{lstlisting}
\vspace{-20pt}

\underline{\textbf{Kết quả:}}

\begin{figure}[H]
    \centering
    \includegraphics[width=0.8\textwidth]{graphics/4_1box.png}
    \caption{Biểu đồ hộp cho các biến số}
    \label{fig:4_1boxplot}
\end{figure}

\underline{\textbf{Nhận xét:}} từ biểu đồ hộp ở hình \ref{fig:4_1boxplot}, ta thấy rằng các biến có một vài ngoại lai nằm ngoài phạm vi của biểu đồ hộp. Bên cạnh đó biểu đồ hợp cũng cho thấy sự khớp như kết quả ở phần trên. Mặc dù vậy các ngoại lai này có thể ảnh hưởng đến các phân tích thống kê tiếp theo, do đó nhóm quyết định loại bỏ các ngoại lai này.
\begin{lstlisting}[language=R, caption=Loại bỏ các ngoại lai khỏi các biến số]
# Dinh nghia ham ngoai lai
is_outlier <- function(x) {
  if (all(is.na(x))) return(rep(FALSE, length(x)))
  
  Q1 <- quantile(x, 0.25, na.rm = TRUE)
  Q3 <- quantile(x, 0.75, na.rm = TRUE)
  IQR <- Q3 - Q1
  
  lower_bound <- Q1 - 1.5 * IQR
  upper_bound <- Q3 + 1.5 * IQR
  
  return(x < lower_bound | x > upper_bound)
}
# Loai bo ngoai lai khoi tap du lieu
outlier_matrix <- sapply(numeric_data, is_outlier)
rows_with_outliers <- rowSums(outlier_matrix) > 0
main_df <- main_df[!rows_with_outliers, ]
\end{lstlisting}
\vspace{-20pt}

% \begin{figure}[H]
%     \centering
%     \includegraphics[width=0.8\textwidth]{graphics/4_1box_no_outlier.png}
%     \caption{Biểu đồ hộp cho các biến số sau khi loại bỏ ngoại lai}
%     \label{fig:4_1box_no_outlier}
% \end{figure}
\subsubsection{Đối với các biến phân loại}
Đối với các biến phân loại như Manufacturer, Memory\_Type, Dedicated, ta có thể đếm số lượng các nhóm con trong các biến này và tần số xuất hiện của chúng như sau:

\begin{lstlisting}[language=R, caption=Đếm số lượng các nhóm con trong các biến phân loại]
table(main_df$Manufacturer)
table(main_df$Memory_Type)
table(main_df$Dedicated)
\end{lstlisting}
\vspace{-20pt}


\underline{\textbf{Kết quả:}}

% \begin{figure}[H]
%     \centering
%     \includegraphics[width=0.8\textwidth]{graphics/4_1dem.png}
%     \caption{Kết quả đếm số lượng các nhóm con trong các biến phân loại}
%     \label{fig:4_2tkmt}
% \end{figure}
\begin{center}
    \begin{tcolorbox}[fontupper=\footnotesize]
        \begin{verbatim}
> table(main_df$Manufacturer)
   AMD    ATI  Intel Nvidia 
   885     76      3   1277 
> table(main_df$Memory_Type)
  DDR2   DDR3  eDRAM  GDDR2  GDDR3  GDDR4  GDDR5 GDDR5X 
    16    356      2      3     93      2   1740     29 
> table(main_df$Dedicated)
  No  Yes 
   3 2238 
        \end{verbatim}
    \end{tcolorbox}
\end{center}

\underline{\textbf{Nhận xét:}} từ kết quả trên, ta có thể thấy được rằng có 4 hãng sản xuất GPU chính trong dữ liệu là Nvidia, AMD, Intel và ATI. Trong đó Nvidia và AMD chiếm đa số, còn Intel và ATI chỉ chiếm một phần rất nhỏ, đặc biệt chỉ có 3 quan sát cho Intel, do đó nhóm quyết định loại bỏ các quan sát của Intel khỏi tập dữ liệu. Ngoài ra, các loại bộ nhớ cũng có sự phân phối không đều, với DDR3, GDDR3, GDDR5 chiếm đa số, các loại bộ nhớ còn lại chỉ chiếm một phần rất nhỏ. Do đó nhóm cũng sẽ loại bỏ các quan sát có loại bộ nhớ là DDR2, eDRAM, GDDR2, GDDR4 khỏi tập dữ liệu. Cuối cùng, biến Dedicated có sự phân phối rất không đều, với chỉ 3 quan sát có giá trị No, do đó nhóm quyết định loại bỏ các quan sát này khỏi tập dữ liệu. Và cũng sẽ không dùng biến Dedicated trong các phân tích tiếp theo.

\begin{lstlisting}[language=R, caption=Loại bỏ các quan sát không phù hợp khỏi tập dữ liệu]
# 1. Tao danh sach ca gia tri can xoa
gia_tri_can_xoa <- c("Intel", "DDR2", "eDRAM", "GDDR2", "GDDR4", "GDDR5X")

# 2. Ham thay the
thay_the_na <- function(df, danh_sach_xoa) {
  df[] <- lapply(df, function(x) {
    replace(x, x %in% danh_sach_xoa, NA)
  })
  return(df)
}

main_df <- thay_the_na(main_df, gia_tri_can_xoa)
main_df <- na.omit(main_df)
\end{lstlisting}
\vspace{-20pt}

\subsubsection{Kiểm tra lại các đặc điểm của các biến.}

Sau khi đã một lần nữa loại bỏ các ngoại lai cũng như quan sát không phù hợp, ta tiến hành kiểm tra lại dữ liệu như sau:

\textbf{Đối với các biến có giá trị số:}

\begin{center}
    \begin{tcolorbox}[fontupper=\footnotesize]
        \begin{verbatim}
> summary(numeric_data)
Memory_Bandwidth    Process       Memory_Speed    Memory_Bus       L2_Cache     
Min.   :  6.4    Min.   :14.00   Min.   : 400   Min.   : 32.0   Min.   :   0.0  
1st Qu.: 64.0    1st Qu.:28.00   1st Qu.:1000   1st Qu.:128.0   1st Qu.: 256.0  
Median :120.0    Median :28.00   Median :1250   Median :192.0   Median : 512.0  
Mean   :135.0    Mean   :30.58   Mean   :1293   Mean   :193.4   Mean   : 690.7  
3rd Qu.:192.3    3rd Qu.:40.00   3rd Qu.:1600   3rd Qu.:256.0   3rd Qu.:1024.0  
Max.   :448.8    Max.   :55.00   Max.   :2127   Max.   :448.0   Max.   :2048.0  
> sds
Memory_Bandwidth    Process       Memory_Speed    Memory_Bus       L2_Cache 
    89.382734      8.879749         382.916229    89.449949      637.044795 
            \end{verbatim}
    \end{tcolorbox}
\end{center}

\begin{figure}[H]
    \centering
    \includegraphics[width=0.65\textwidth]{graphics/4_1box_no_outlier.png}
    \caption{Biểu đồ hộp cho các biến số sau khi loại bỏ ngoại lai và quan sát không phù hợp}
    \label{fig:4_2boxplot_final}
\end{figure}

\textbf{\underline{Nhận xét:}} sau khi loại bỏ các ngoại lai và giá trị không phù hợp, ta thấy rằng phương sai của các biến đã giảm đáng kể, ngoài ra trong biểu đồ hộp, các ngoại lai cũng đã không còn, khiến cho biểu đồ hộp trực quan hơn. Bênh cạnh đó việc loại bỏ các ngoại lai và quan sát không phù hợp cũng giúp cho các phân tích thống kê tiếp theo trở nên chính xác hơn.

\textbf{Đối với các biến phân loại:}

Như đã đề cập ở trên vì biến Dedicated đã bị loại bỏ khỏi tập dữ liệu, ta chỉ kiểm tra lại hai biến còn lại là Manufacturer và Memory\_Type như sau:

\begin{center}
    \begin{tcolorbox}[fontupper=\footnotesize]
        \begin{verbatim}
> table(main_df$Manufacturer)
   AMD    ATI Nvidia 
   876     76   1236 
> table(main_df$Memory_Type)
 DDR3 GDDR3 GDDR5 
  355    93  1740 
        \end{verbatim}
    \end{tcolorbox}
\end{center}

\textbf{\underline{Nhận xét:}} như vậy ta chỉ còn lại 3 hãng sản xuất GPU chính là Nvidia, AMD và ATI. Bên cạnh đó ta cũng chỉ còn lại 3 loại bộ nhớ chính là DDR3, GDDR3 và GDDR5.

\subsection{Sự phân phối tần số của biến Memory\_Bandwidth.}

Sự phân phối tần số của biến Memory\_Bandwidth được thể hiện qua biểu đồ tần số sau:

\begin{figure}[H]
    \centering
    \includegraphics[width=0.8\textwidth]{graphics/4_1bwhist.png}
    \caption{Biểu đồ tần số của biến Memory\_Bandwidth}
    \label{fig:4_3hist}
\end{figure}

\underline{\textbf{Nhận xét:}} từ biểu đồ tần số ở hình \ref{fig:4_3hist}, ta thấy rằng sự phân phối của biến Memory\_Bandwidth của các GPU trong tập dữ liệu. Biến không tuân theo phân phối chuẩn, phân phối lệch phải, biến có phần lớn giá trị tập trung thấp, chủ yếu ở ngưỡng dưới 250 GB/s và có xu hướng giảm dần khi giá trị của “Memory\_Bandwidth” tăng lên. Điều này phản ánh nhu cầu thị trường đối với các loại GPU. Các trường hợp có băng thông bộ nhớ cao là rất hiếm, có sự chênh lệch lớn giữa các đối tượng quan sát được, cho thấy một số loại GPU hiệu năng cao có khả năng được sử dụng cho các công việc đặc thù. Do đó khi phân tích thống kê chỉ phù hợp khi xemm xét các GPU có hiệu năng trung bình, phổ thông.

% biến Memory\_Bandwidth có phân phối lệch phải, với phần lớn các giá trị tập trung ở bên trái biểu đồ. Điều này cho ta thấy rằng biến Memory\_Bandwidth không tuân theo phân phối chuẩn.


\subsection{Đồ thị scatter plot cho biến Memory\_Bandwidth theo Memory\_Bus, L2\_Cache, Memory\_Speed, Process và Memory.}

Ta tiến hành vẽ các đồ thị scatter plot để quan sát mối quan hệ giữa biến Memory\_Bandwidth với các biến Memory\_Bus, L2\_Cache, Memory\_Speed, Process như sau:

\begin{lstlisting}[language=R, caption=Vẽ đồ thị scatter plot.]
ve_bieu_do_tuy_chinh <- function(du_lieu_x, du_lieu_y, nhan_x, nhan_y) {
  plot(x = du_lieu_x, 
       y = du_lieu_y,
       main = paste("Moi quan he giua", nhan_x, "va", nhan_y),
       xlab = nhan_x,
       ylab = nhan_y,
       pch = 19,          # Style
       col = "darkblue",  # Color
       frame = FALSE)     # Remove extra frame
}
\end{lstlisting}
\vspace{-20pt}

\underline{\textbf{Kết quả:}}

\begin{figure}[H]
    \centering
    \includegraphics[width=0.8\textwidth]{graphics/4_1scatter.png}
    \caption{Đồ thị scatter plot cho biến Memory\_Bandwidth theo Memory\_Bus, L2\_Cache, Memory\_Speed, Process}
    \label{fig:4_4scatter}
\end{figure}

\underline{\textbf{Nhận xét:}}

\begin{itemize}
    \item Dựa vào đồ thị scatter plot cho Memory\_Bandwidth theo Memory\_Bus, nhóm thấy được rằng Memory\_Bandwidth phân bố rời rạc và không đồng đều, tập trung chủ yếu ở các mức Memory\_Bus phổ biến như 128 bit, 192 bit, và 256 bit. Mối quan hệ giữa hai biến này không rõ ràng, với sự phân tán dữ liệu rộng ở mỗi mức Memory\_Bus. Điều này cho thấy rằng Memory\_Bus không phải là yếu tố duy nhất quyết định băng thông bộ nhớ, mà còn phụ thuộc vào các yếu tố khác như loại bộ nhớ và tốc độ bộ nhớ.
    \item Dựa vào đồ thị scatter plot cho Memory\_Bandwidth theo L2\_Cache, nhóm thấy được rằng phần lớn các điểm dữ liệu tập trung ở vùng Cache nhỏ, dưới 1000KB. Mối quan hệ giữa Memory\_Bandwidth và L2\_Cache không rõ ràng, với sự phân tán dữ liệu rộng ở mỗi mức L2\_Cache.
    \item Dựa vào đồ thị scatter plot cho Memory\_Bandwidth theo Memory\_Speed, nhóm thấy được rằng đây là biến có ảnh hưởng lớn nhất đền Memory\_Bandwidth vì dữ liệu phân bố theo đường chéo. Mặc dù vậy, phần lớn dữ liệu vẫn tập trung ở vùng Memory\_Speed thấp, dưới 1500MHz.
    \item Dựa vào đồ thì scatter plot cho Memory\_Bandwidth theo Process, nhóm thấy được rằng phần lớn các điểm dữ liệu tập trung ở chủ yếu ở 3 mức Process cụ thể là 28mm, 40mm và 55mm. Mối quan hệ giữa hai biến này không rõ ràng, với sự phân tán dữ liệu rộng ở mỗi mức Process.
    \item Dựa vào đồ thị scatter plot cho Memory\_Bandwidth theo Memory, nhóm thấy được rằng phần lớn các điểm dữ liệu phân bố không đều, tập trung phân bố ở vùng nhỏ hơn 10000MB.
\end{itemize}

\subsection{Ma trận tương quan cho các biến số.}

Ta tiến hành tính ma trận tương quan và vẽ ma trận tương quan cho các biến số như sau:

\begin{lstlisting}[language=R, caption=Tính ma trận tương quan và vẽ ma trận tương quan cho các biến số]
### Ve ma tran tuong quan ###
library(corrplot)
# 1. Chon cac bien can ve ma tran tuong quan
my_data <- main_df[, c("Memory_Bandwidth", "Process", "Memory_Speed", "Memory_Bus", "L2_Cache", "Memory")]

# 2. Tinh toan ma tran tuong quan
cor_matrix <- cor(my_data, use = "complete.obs")
corrplot(cor_matrix, 
         method = "circle",       # Kieu hien thi: hinh tron
         type = "upper",          # Chi ve mot nua tren (cho do roi)
         order = "hclust",        # Tu dong gom nhom cac bien giong nhau lai gan nhau
         tl.col = "black",        # MMau chu ten bien
         tl.srt = 45,             # Nghieng chu 45 do cho de doc
         addCoef.col = "black",   # Hien thi them con so cu the (QUAN TRONG)
         number.cex = 0.8         # Co chu so
)
\end{lstlisting}

\textbf{\underline{Kết quả:}}
\begin{figure}[H]
    \centering
    \includegraphics[width=0.8\textwidth]{graphics/4_1corr.png}
    \caption{Ma trận tương quan cho các biến số}
    \label{fig:4_5correlation}
\end{figure}

\textbf{\underline{Nhận xét:}} từ ma trận tương quan ở hình \ref{fig:4_5correlation}, ta thấy được rằng các biến Memory\_Bus, Memory\_Speed và Memory có mối quan hệ tương quan mạnh với Memory\_Bandwidth, với hệ số tương quan lần lượt là 0.76, 0.68, 0.63. Điều này cho thấy rằng các biến này có ảnh hưởng lớn đến băng thông bộ nhớ của GPU. Bên cạnh đó 2 biến Process và Memory\_Speed cũng có mối quan hệ tương quan âm mạnh với nhau. Tựu trung lại, các biến này đều có mối quan hệ tương quan với nhau, cho thấy rằng chúng có thể ảnh hưởng đến hiệu suất xử lý của GPU.