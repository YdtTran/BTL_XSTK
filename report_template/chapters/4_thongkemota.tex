\section{Thống kê mô tả}

Sau khi đã loại bỏ các các biến có tỉ lệ dữ liệu khuyết cao, nhóm tiến hành chọn lọc các biến quan trọng. Sau quá trình tham khảo, nhóm nhận thấy thông số quyết định hiệu suất xử lí của GPU là băng thông bộ nhớ (Memory\_Bandwidth) vì nó quyết định tốc độ truyền tải dữ liệu giữa bộ nhớ và GPU, đặc biệt với các tác vụ yêu cầu GPU xử lí lượng dữ liệu lớn. Băng thông càng cao thì tốc độ xử lý và khả năng đa nhiệm của hệ thống càng nhanh và ổn định. Ngoài ra, nhóm lựa chọn các biến có tầm ảnh hưởng khác nhau đến Memory\_Bandwidth gồm: Manufacturer, Process, Memory\_Speed, Memory\_Bus, Memory\_Type, L2\_Cache, Dedicated.

\subsection{Tính các giá trị đặc trưng của các biến.}

\subsubsection{Đối với các biến có giá trị số.}
Ta tiến hành lấy các biến trong dữ liệu mẫu ra để tính các giá trị đặc trưng của các biến như sau:

\begin{lstlisting}[language=R, caption=Tính các giá trị đặc trưng của các biến]
numeric_data <- main_df[, c("Process", "Memory_Speed", "Memory_Bus" , "L2_Cache")]
summary(numeric_data)
\end{lstlisting}

\underline{\textbf{Kết quả:}}

\begin{figure}[H]
    \centering
    \includegraphics[width=0.8\textwidth]{graphics/4_1tkmt.png}
    \caption{Kết quả tính các giá trị đặc trưng của các biến}
    \label{fig:4_1tkmt}
\end{figure}

\underline{\textbf{Nhận xét:}} từ kết quả ở hình \ref{fig:4_1tkmt}, ta có thể xem giá trị lớn nhất, nhỏ nhất của các biến cũng như giá trị trung bình của chúng.

Nhóm cũng thử vẽ biểu đồ hộp để trực quan hóa trực quan hoá các số liệu trong hình \ref{fig:4_1tkmt}:
\begin{lstlisting}[language=R, caption=Vẽ biểu đồ hộp cho các biến số]
par(mfrow = c(2, 2)) # Chia khung thanh 2 hang va 2 cot

boxplot(main_df$Process, main = "Process (nm)", col = "lightblue", ylab = "Value")
boxplot(main_df$Memory_Speed, main = "Memory Speed (MHz)", col = "lightgreen", ylab = "Value")
boxplot(main_df$Memory_Bus, main = "Memory Bus (Bit)", col = "pink", ylab = "Value")
boxplot(main_df$L2_Cache, main = "L2 Cache (KB)", col = "wheat", ylab = "Value")

par(mfrow = c(1, 1)) # Chinh lai nhu ban dau

\end{lstlisting}

\underline{\textbf{Kết quả:}}

\begin{figure}[H]
    \centering
    \includegraphics[width=0.8\textwidth]{graphics/4_1box.png}
    \caption{Biểu đồ hộp cho các biến số}
    \label{fig:4_1boxplot}
\end{figure}

\underline{\textbf{Nhận xét:}} từ biểu đồ hộp ở hình \ref{fig:4_1boxplot}, ta thấy rằng các biến có một vài ngoại lai nằm ngoài phạm vi của biểu đồ hộp. Bên cạnh đó biểu đồ hợp cũng cho thấy sự khớp như kết quả ở phần trên.


\subsubsection{Đối với các biến phân loại}
Đối với các biến phân loại như Manufacturer, Memory\_Type, Dedicated, ta có thể đếm số lượng các nhóm con trong các biến này và tần số xuất hiện của chúng như sau:

\begin{lstlisting}[language=R, caption=Đếm số lượng các nhóm con trong các biến phân loại]
table(main_df$Manufacturer)
table(main_df$Memory_Type)
table(main_df$Dedicated)
\end{lstlisting}


\underline{\textbf{Kết quả:}}

\begin{figure}[H]
    \centering
    \includegraphics[width=0.8\textwidth]{graphics/4_1dem.png}
    \caption{Kết quả đếm số lượng các nhóm con trong các biến phân loại}
    \label{fig:4_2tkmt}
\end{figure}

\underline{\textbf{Nhận xét:}} từ kết quả ở hình \ref{fig:4_2tkmt}, ta có thể thấy được rằng có 4 hãng sản xuất GPU chính trong dữ liệu là Nvidia, AMD, Intel và ATI. Trong đó Nvidia và AMD chiếm đa số, còn Intel và ATI chỉ chiếm một phần rất nhỏ. Ngoài ra, ta có 11 loại bộ nhớ khác nhau. Về Dedicated ta chỉ có 2 giá trị là Yes và No.


\subsection{Sự phân phối tần số của biến Memory\_Bandwidth.}

Sự phân phối tần số của biến Memory\_Bandwidth được thể hiện qua biểu đồ tần số sau:

\begin{figure}[H]
    \centering
    \includegraphics[width=0.8\textwidth]{graphics/4_1bwhist.png}
    \caption{Biểu đồ tần số của biến Memory\_Bandwidth}
    \label{fig:4_3hist}
\end{figure}

\underline{\textbf{Nhận xét:}} từ biểu đồ tần số ở hình \ref{fig:4_3hist}, ta thấy rằng sự phân phối của biến Memory\_Bandwidth của các GPU trong tập dữ liệu. Biến không tuân theo phân phối chuẩn, phân phối lệch phải, biến có phần lớn giá trị tập trung thấp, chủ yếu ở ngưỡng dưới 200 GB/s và có xu hướng giảm dần khi giá trị của “Memory\_Bandwidth” tăng lên. Điều này phản ánh nhu cầu thị trường đối với các loại GPU. Các trường hợp có băng thông bộ nhớ cao là rất hiếm, có sự chênh lệch lớn giữa các đối tượng quan sát được, cho thấy một số loại GPU hiệu năng cao có khả năng được sử dụng cho các công việc đặc thù.

% biến Memory\_Bandwidth có phân phối lệch phải, với phần lớn các giá trị tập trung ở bên trái biểu đồ. Điều này cho ta thấy rằng biến Memory\_Bandwidth không tuân theo phân phối chuẩn.


\subsection{Đồ thị scatter plot cho biến Memory\_Bandwidth theo Memory\_Bus, L2\_Cache, Memory\_Speed, Process.}

Ta tiến hành vẽ các đồ thị scatter plot để quan sát mối quan hệ giữa biến Memory\_Bandwidth với các biến Memory\_Bus, L2\_Cache, Memory\_Speed, Process như sau:
\begin{lstlisting}[language=R, caption=Vẽ đồ thị scatter plot.]
ve_bieu_do_tuy_chinh <- function(du_lieu_x, du_lieu_y, nhan_x, nhan_y) {
  plot(x = du_lieu_x, 
       y = du_lieu_y,
       main = paste("Moi quan he giua", nhan_x, "va", nhan_y),
       xlab = nhan_x,
       ylab = nhan_y,
       pch = 19,          # Style
       col = "darkblue",  # Color
       frame = FALSE)     # Remove extra frame
}
\end{lstlisting}

\underline{\textbf{Kết quả:}}

\begin{figure}[H]
    \centering
    \includegraphics[width=0.8\textwidth]{graphics/4_1scatter.png}
    \caption{Đồ thị scatter plot cho biến Memory\_Bandwidth theo Memory\_Bus, L2\_Cache, Memory\_Speed, Process}
    \label{fig:4_4scatter}
\end{figure}

\underline{\textbf{Nhận xét:}}

\begin{itemize}
    \item Dựa vào đồ thị scatter plot cho Memory\_Bandwidth theo Memory\_Bus, nhóm thấy được rằng phần lớn điểm dữ liệu tập trung ở vùng Memory\_Bus nhỏ dưới 1000 bit, với băng thông không quá 500 GB/s. Điều này cho thấy đây là là các Bus bộ nhớ phổ biến cho các hệ thống bộ nhớ. Mối quan hệ phụ thuộc giữa Memory Bandwidth và Memory Bus chỉ ở mức trung bình. Có sự phân tán rất lớn trong băng thông bộ nhớ, cho thấy rằng ngoài Memory Bus còn nhiều yếu tố khác ảnh hưởng đến Memory\_Bandwidth.
    \item Dựa vào đồ thị scatter plot cho Memory\_Bandwidth theo L2\_Cache, nhóm thấy được rằng phần lớn các điểm dữ liệu tập trung ở vùng Cache nhỏ, dưới 2000 KB, với băng thông bộ nhớ từ thấp đến trung bình. Từ đây cho thấy đa số hệ thống trong tập dữ liệu có dung lượng L2\_Cache khiêm tốn. Ở vùng L2 Cache lớn hơn, trên mức 4000 KB, băng thông bộ nhớ có độ phân tán rất rộng, từ mức trung bình đến cao.
    \item Dựa vào đồ thị scatter plot cho Memory\_Bandwidth theo Memory\_Speed, nhóm thấy được rằng Phần lớn dữ liệu tập trung ở vùng bộ nhớ thấp, dưới 1000 MHz, băng thông trải dài từ thấp đến trung bình. Giữa Memory\_Bandwidth và Memory\_Speed có mối quan hệ phụ thuộc rõ ràng, dù ở băng thông cao hơn vẫn có sự phân tán dữ liệu khá lớn. Khi tốc độ bộ nhớ tăng lên thì băng thông bộ nhớ cũng có xu hướng tăng dần, điều này cho thấy Memory\_Speed có ảnh hưởng mạnh mẽ đến Memory\_Bandwidth.


\end{itemize}