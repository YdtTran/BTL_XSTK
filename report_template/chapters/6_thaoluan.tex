\section{Thảo luận và mở rộng}

\subsection{Thảo luận}

Bài báo cáo đã cung cấp cơ sở dữ liệu và phân tích các thuộc tính về GPU cho ta thấy cái nhìn bao quát và hiểu rõ hơn các yếu tố ảnh hưởng đến hiệu suất của GPU. Các mô hình được xây dựng để dự đoán có thể đánh giá hiệu năng GPU thông qua các thông số kỹ thuật. Sau khi phân tích, ta có một số nhận xét sau:


\begin{itemize}
    \item Thống kê mô tả:
          \begin{itemize}
              \item Cho thấy sự phân bố không đồng đều của các thông số, đặc biệt là  Memory\_Bandwidth-thông số quyết định hiệu suất xử lý của GPU phần lớn giá trị tập trung ở mức thấp.
              \item Các biến số như Memory\_Speed, Memory\_Bus, L2\_Cache có ảnh hưởng nhất định đến Memory\_Bandwidth, tuy nhiên mức độ ảnh hưởng khác nhau và có sự phân tán lớn trong dữ liệu.
              \item Trong dữ liệu chỉ gồm có một số ít nhà sản xuất chiếm đa số, điều này có thể ảnh hưởng đến tính tổng quát của mô hình dự đoán.
              \item Dữ liệu có một số ngoại lai, điều này có thể ảnh hưởng đến trực tiếp đến kết quả phân tích và các mô hình dự đoán.
          \end{itemize}

    \item Thống kê suy diễn:

          \begin{itemize}
              \item Mô hình hồi quy tuyến tính đa biến cho thấy các biến Process, Memory, Memory\_Speed, Memory\_Bus, L2\_Cache có ảnh hưởng đáng kể đến Memory\_Bandwidth, tuy nhiên mô hình có R-squared không cao, chỉ khoảng 0.8419, do đó đây chưa phải là mô hình hồi quy tốt nhất.
              \item Như đã đề cập ở trên, phần lớn các biến xét đến trong các mô hình và phân tích đều không tuân theo phân phối chuẩn, điều này ảnh hưởng đến tính chính xác của các mô hình dự đoán, buộc nhóm phải tìm các giải pháp để cải thiện các mô hình đó.
          \end{itemize}
\end{itemize}


\subsection{Mở rộng}

\textbf{Cải thiện mô hình hồi quy tuyến tính bội.}

Như đã đề cập ở phần thống kê suy diễn, mô hình hồi quy tuyến tính bội hiện tại có R-squared không cao, chỉ khoảng 0.8419 vì biến Memory\_Bandwidth không tuân theo phân phối chuẩn. Do đó Ta có thể thử dùng phép biết đổi logarit ($\log$) cho các biến để ổn định phương sai của các sai số, làm cho độ phân tán của sai số giảm xuống, phương sai trở nên đồng nhất hơn. Vì trong dữ liệu có giá trị 0 ở một số biến (VD: L2\_Cache) nên ta sẽ lấy $\log(X+1)$ để tránh giá trị lỗi. Ta thực hiện như sau:

\begin{lstlisting}[language=R, caption=Phép biến đổi logarit cho các biến.]
training_set_log <- training_set
testing_set_log <- testing_set
training_set_log$Memory <- log(training_set_log$Memory + 1)
training_set_log$Memory_Bus <- log(training_set_log$Memory_Bus + 1)
training_set_log$Memory_Bandwidth <- log(training_set_log$Memory_Bandwidth +
                                           1)
training_set_log$Memory_Speed <- log(training_set_log$Memory_Speed + 1)
training_set_log$L2_Cache <- log(training_set_log$L2_Cache + 1)
training_set_log$Process <- log(training_set_log$Process + 1)
testing_set_log$Memory <- log(testing_set_log$Memory + 1)
testing_set_log$Memory_Bus <- log(testing_set_log$Memory_Bus + 1)
testing_set_log$Memory_Bandwidth <- log(testing_set_log$Memory_Bandwidth +
                                          1)
testing_set_log$Memory_Speed <- log(testing_set_log$Memory_Speed + 1)
testing_set_log$L2_Cache <- log(testing_set_log$L2_Cache + 1)
testing_set_log$Process <- log(testing_set_log$Process + 1)
new_model <- lm(
  Memory_Bandwidth ~ Process + Memory + Memory_Speed
  + Memory_Bus + L2_Cache,
  training_set_log
)
summary(new_model)
\end{lstlisting}

\underline{\textbf{Kết quả:}}

\begin{figure}[H]
    \centering
    \includegraphics[width=0.8\textwidth]{graphics/KQ2.png}
    \caption{Kết quả mô hình hồi quy tuyến tính bội sau khi biến đổi logarit}
    \label{fig:6_1model}
\end{figure}

\underline{\textbf{Kết luận:}}

Mặc dù mô hình có thật sự cải thiện hơn so với trước với R-squared là 0.8874, tuy nhiên p\_value của biến Process lớn hơn 0.05, nên ta sẽ loại bỏ biến này trong mô hình mới.

Xây dựng lại mô hình mới được kết quả như sau:

\begin{figure}[H]
    \centering
    \includegraphics[width=0.6\textwidth]{graphics/KQ3.png}
    \caption{Kết quả mô hình hồi quy tuyến tính bội sau khi loại bỏ biến Process}
    \label{fig:6_2model}
\end{figure}

\underline{\textbf{Nhận xét:}} có thể dễ dàng nhận thấy, tất cả các chỉ số đã có sự cải thiện rõ rệt so với các mô hình trước. Cụ thể R-squared đã tăng lên 0.9067. Do đó, mô hình hồi quy hiện tại hoạt động hiệu quả, chính xác và đáng tin cậy hơn.

\textbf{Dự đoán giá trị Memory\_Bandwidth sử dụng mô hình mới.}

\begin{lstlisting}[language=R, caption=Dự đoán giá trị Memory\_Bandwidth sử dụng mô hình mới.]
pred_log_values <- predict(new_model, newdata = testing_set_log)

# Chuyen tu log ve gia tri binh thuong
pred_original_scale <- exp(pred_log_values) - 1

# Lay gia tri thuc te
actual_original_scale <- testing_set$Memory_Bandwidth

# Tao df cho de nhin
results_summary_display_original <- data.frame(
  Memory_Bandwidth_Actual = actual_original_scale,
  Memory_Bandwidth_Predict = pred_original_scale,
  Error_Original = actual_original_scale - pred_original_scale # Tinh sai so
)

head(results_summary_display_original, 10)
\end{lstlisting}

\underline{\textbf{Kết quả:}}

\begin{figure}[H]
    \centering
    \includegraphics[width=0.8\textwidth]{graphics/Test.png}
    \caption{Kết quả dự đoán giá trị Memory\_Bandwidth sử dụng mô hình mới}
    \label{fig:6_2predict}
\end{figure}

\begin{figure}[H]
    \centering
    \includegraphics[width=0.8\textwidth]{graphics/6_1better.png}
    \caption{Đồ thị giữa giá trị thực tế và dự đoán của Memory\_Bandwidth (mô hình cải tiến)}
    \label{fig:6_3error}
\end{figure}

\begin{figure}[H]
    \centering
    \includegraphics[width=0.8\textwidth]{graphics/rmse_better.png}
    \caption{MRSE và RR của mô hình cải tiến}
    \label{fig:mrse_rr_better}
\end{figure}

\underline{\textbf{Nhận xét:}}

Có thể thấy rằng mô hình sau khi biến đổi logarit đã cải thiện đáng kể độ chính xác trong việc dự đoán giá trị Memory\_Bandwidth. Đồ thị ở hình \ref{fig:6_3error} cho thấy các điểm dữ liệu tập trung gần đường chéo $y=x$ hơn so với mô hình gốc ở hình \ref{fig:kq2_hqtt}. Mặc dù vậy khi tính ngược lại giá trị gốc (bằng cách lấy hàm mũ), ta thấy rằng MRSE của mô hình mới là 41.0669 cao hơn so với mô hình gốc (33.2738), mặc dù vậy RR của mô hình mới lại cao hơn so với mô hình gốc.