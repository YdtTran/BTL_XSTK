% Speed-up compilation
\documentclass[final, twoside]{hcmut-report-vi}
\usepackage{codespace}

% Draft watermark
% https://github.com/callegar/LaTeX-draftwatermark

% Encodings
\usepackage{gensymb,textcomp}
% \usepackage{subfigure}
\usepackage{subcaption}
% Better tables
% Wide tables go to https://tex.stackexchange.com/q/332902
\usepackage{array,longtable,multicol,multirow,siunitx,tabularx}

\usepackage{tcolorbox}


% Better enum
\usepackage{enumitem}

% Graphics
\usepackage{caption,float}

% Add options for figures, like max width, framing, etc.
\usepackage[export]{adjustbox}

% References
% Use \Cref{} instead of \ref{}
\usepackage[nameinlink]{cleveref}

% FOR DEMONSTRATION PURPOSES, REMOVE IN PRODUCTION
\usepackage{mwe}

\newcommand{\blankpage}{
    \clearpage            % Đẩy tất cả nội dung/hình ảnh còn sót lại ra trang trước
    \thispagestyle{empty} % Xóa số trang, header, footer của trang này
    \null                 % Thêm một ký tự rỗng (để LaTeX biết trang này có tồn tại)
    \newpage              % Kết thúc trang trắng và sang trang tiếp theo
}

% Sub-preambles
% https://github.com/MartinScharrer/standalone

% === THÊM CODE NÀY ĐỂ GIẢM KHOẢNG CÁCH CÔNG THỨC ===
% =================================================

\onehalfspacing

% Configurations
\coursename{Xác suất thống kê MT2013}
\reporttype{Báo cáo bài tập lớn}
\title{Phân tích dữ liệu GPU}
\advisor{& \textbf{Nguyễn Kiều Dung} &}
% \stuname{%
%   & Lê Ngô Khôi Nguyên  & 2412338 & L07\\
%   & Phan Nguyễn Hoàng Minh  & 2412114 & L07\\
%   & \textbf{Trần Đình Ý}  & \textbf{2414100} & \textbf{L07}\\
%   & Phạm Nghĩa Trung & 2413713 & L08\\
%   & Hà Huy Toàn & 2413525 & L09\\
%   & Nguyễn Gia Hưng & 2411356 & L09\\
%   & Nguyễn Quốc Trung & 2413707 & L14\\
%   & Phạm Minh Đức & 2414106 & L15\\
% }



% ===================================================
% NỘI DUNG CHÍNH
% ===================================================
\begin{document}

\renewcommand{\figurename}{Hình}
\renewcommand{\tablename}{Bảng}
\renewcommand{\listfigurename}{Danh sách hình}
\renewcommand{\listtablename}{Danh sách bảng}
\renewcommand{\contentsname}{Mục lục}
\renewcommand{\refname}{Tài liệu tham khảo}

\coverpage
\begin{center}
  \section*{Danh sách thành viên}
  % Định nghĩa 6 cột: 
  % c (STT) | l (Tên) | c (MSSV) | c (Lớp) | m{5cm} (Vai trò - tự xuống dòng) | c (Điểm)
  \begin{tabular}{|c|l|c|c|m{5cm}|c|}
    \hline
    \textbf{STT} & \textbf{Họ và tên}     & \textbf{MSSV}    & \textbf{Lớp} & \textbf{Vai trò}                      & \textbf{Điểm} \\
    \hline
    1            & Lê Ngô Khôi Nguyên     & 2412338          & L07          & Thống kê mô tả                        &               \\
    \hline
    2            & Phan Nguyễn Hoàng Minh & 2412114          & L07          & Anova một yếu tố                      &               \\
    \hline
    3            & \textbf{Trần Đình Ý}   & \textbf{2414100} & \textbf{L07} & Tổng hợp bài làm                      &               \\
    \hline
    4            & Phạm Nghĩa Trung       & 2413713          & L08          & Tổng quan dữ liệu và kiến thức nền    &               \\
    \hline
    6            & Nguyễn Gia Hưng        & 2411356          & L09          & Hồi quy tuyến tính bội                &               \\
    \hline
    7            & Nguyễn Quốc Trung      & 2413707          & L14          & Kiểm định 2 mẫu, Thảo luận và mở rộng &               \\
    \hline
    8            & Phạm Minh Đức          & 2414106          & L15          & Kiểm định 1 mẫu                       &               \\
    \hline
  \end{tabular}
\end{center}
\blankpage

\newpage
% Tạo mục lục và các danh sách
\pagestyle{empty}
\tableofcontents
\pagestyle{empty}
\newpage
\listoffigures
\pagestyle{empty}
\newpage
\listoftables
\pagestyle{empty}
\newpage
\lstlistoflistings
\pagestyle{empty}
\clearpage

% Nội dung các chương

\pagestyle{fancy}
\setcounter{page}{1}
\newpage
% display python code
\input{chapters/1_tongquandulieu.tex}
\newpage
\section{Kiến thức nền}

\subsection{Thống kê mô tả và thống kê suy diễn}

\textbf{Thống kê mô tả (descriptive statistics):} là quá trình thu thập, biểu diễn, tổng hợp và xử lý dữ liệu để biến đổi dữ liệu thành thông tin.

\textbf{Thống kê suy diễn (Inferential statistics):} xử lý các thông tin có được từ thống kê mô tả, từ đó đưa ra các cơ sở cho những dự đoán (predictions), dự báo (forecasts) và các ước lượng (estimations).
\subsection{Các đặc trưng của tổng thể và mẫu}

\subsubsection{Khái niệm}

\textbf{Tổng thể thống kê (population):} là tập hợp các phần tử thuộc đối tượng nghiên cứu, cần được quan sát, thu thập và phân tích theo một hoặc một số đặc trưng nào đó. Các phần tử tạo thành tổng thể thống kê được gọi là đơn vị tổng thể.

\textbf{Mẫu (sample):} là một số đơn vị được chọn ra từ tổng thể theo một phương pháp lấy mẫu nào đó. Các đặc trưng mẫu được sử dụng để suy rộng ra các đặc trưng của tổng thể nói chung.

\textbf{Đặc điểm thống kê (dấu hiệu nghiên cứu):} là các tính chất quan trọng liên quan trực tiếp đến nội dung nghiên cứu và khảo sát cần thu thập dữ liệu trên các đơn vị tổng thể. Người ta chia làm 2 loại: đặc điểm thuộc tính và đặc điểm số lượng.

\subsubsection{Tỷ lệ}

Với một tổng thể có N phần tử và M phần tử mang tính chất A nào đó. Tỷ lệ tổng thể (kí hiệu: p) được tính bởi công thức:
\[ p = \frac{M}{N} \]

Với một mẫu có n phần tử và có m phần tử mang tính chất A nào đó. Tỷ lệ mẫu (kí hiệu: f hay $\bar{p}$) được tính bởi công thức:
\[ p = \bar{f} = \frac{m}{n} \]

\subsubsection{Trung bình}

\textbf{Trung bình (mean):} là đại lượng thường được sử dụng nhất để đo giá trị trung tâm của dữ liệu (Trung bình bị ảnh hưởng bởi các giá trị ngoại lai). Với một tổng thể có $N$ phần tử, trung bình tổng thể (kí hiệu: $\mu$ hay $\bar{X}$) tính bởi công thức:
\[ \mu = \frac{1}{N} \sum_{i=1}^{N} X_i = \frac{X_1 + X_2 + \dots + X_N}{N} \]

Với một mẫu có $n$ phần tử, trung bình mẫu (kí hiệu: $\bar{x}$) tính bởi công thức:
\[ \bar{x} = \frac{1}{n} \sum_{i=1}^{n} x_i = \frac{x_1 + x_2 + \dots + x_n}{n} \]

Trong trường hợp X có bảng phân phối tần số như sau:

\begin{center}
    \begin{tabular}{|c|c|c|c|c|c|}
        \hline
        \textbf{X}      & $x_1$ & $x_2$ & $x_3$ & $\dots$ & $x_k$ \\ \hline
        \textbf{Tần số} & $n_1$ & $n_2$ & $n_3$ & $\dots$ & $n_k$ \\ \hline
    \end{tabular}
\end{center}

Ta lại có trung bình mẫu tính bởi công thức:
\[ \bar{x} = \frac{1}{n}\displaystyle\sum_{i=1}^{k} x_i n_i = \frac{x_1 n_1 + x_2 n_2 + \dots + x_k n_k}{n} \]

\subsubsection{Phương sai, độ lệch chuẩn}

\textbf{Phương sai (Variance):} là trung bình của bình phương độ lệch các giá trị so vói trung bình. Phương sai phản ánh độ phân tán hay sự biến thiên của dữ liệu.

\textbf{Độ lệch chuẩn (Standard deviation):} Độ lệch chuẩn (Standard deviation) là căn bậc hai của phương sai. Độ lệch chuẩn dùng để đo sự biến thiên, biểu diễn sự biến thiên xung quanh trung bình và cũng có cùng đơn vị đo với dữ liệu gốc.

Với một tổng thể có N phần tử, phương sai tổng thể (kí hiệu: $\sigma^2$) tính bởi công thức:
\[ \sigma^2 = \frac{1}{N}\sum_{i=1}^{N} (x_i - \mu)^2\]

Khi đó: $\sigma$ được gọi là độ lệch chuẩn của tổng thể.

Với một mẫu có $n$ phần tử, phương sai mẫu (kí hiệu: $s^2$) tính bởi công thức:
\[ s^2 = \frac{1}{n-1}\sum_{i=1}^{n} (x_i - \bar{x})^2 \]

Trong trường hợp X có bảng phân phối tần số như sau:

\begin{center}
    \begin{tabular}{|c|c|c|c|c|c|}
        \hline
        \textbf{X}      & $x_1$ & $x_2$ & $x_3$ & $\dots$ & $x_k$ \\ \hline
        \textbf{Tần số} & $n_1$ & $n_2$ & $n_3$ & $\dots$ & $n_k$ \\ \hline
    \end{tabular}
\end{center}

Ta lại có phương sai mẫu tính bởi công thức:
% Ghi chú: Hình ảnh của bạn có một lỗi nhỏ (n/n-1), tôi đã sửa lại theo công thức chuẩn.
\[ s^2 = \frac{1}{n-1} \sum_{i=1}^{k} n_i (x_i - \bar{x})^2\]

Khi đó: $s$ được gọi là độ lệch chuẩn    mẫu.

\subsubsection{Các đặc trưng khác}

\textbf{Yếu vị (Mode):} là giá trị của phần tử có số lần xuất hiện lớn nhất trong mẫu. Yếu vị không bị ảnh hưởng bởi các điểm ngoại lai.

\textbf{Hệ số biến thiên (Coefficient of variation):} đo lường mức độ biến động tương đối của mẫu dữ liệu, được dùng khi người ta muốn so sánh mức độ biến động của các mẫu không cùng đơn vị đo. Đơn vị tính bằng \%.
\[ CV(\text{tongthe}) = \frac{\sigma}{\mu} \times 100\% \]
\[ CV(\text{mau}) = \frac{s}{\bar{x}} \times 100\% \]

\textbf{Sai số chuẩn (Standard Error):} là giá trị đại diện cho độ lệch chuẩn của giá trị trung bình trong tập dữ liệu. Nó phục vụ như một thước đo biến động cho các biến ngẫu nhiên hay độ lượn độ phân tán. Độ phân tán càng nhỏ, dữ liệu càng chính xác.
% \[ SE(\text{tongthe}) = \frac{\sigma}{\sqrt{N}} \]
% \[ SE(\text{mau}) = \frac{s}{\sqrt{n}} \]

\textbf{Trung vị (Median):} Giả sử $X$ có $N$ quan sát, sắp các quan sát này theo thứ tự tăng dần. Trung vị là giá trị nằm chính giữa dãy số này và chia nó thành 2 phần bằng nhau. Cụ thể:

Giả sử mẫu có kích thước $n$ được sắp xếp tăng dần theo giá trị được khảo sát:
\[ x_1 \le x_2 \le \dots \le x_{n-1} \le x_n \]

Nếu $n = 2k + 1$ (n lẻ) thì trung vị mẫu là giá trị $x_{k+1}$

Nếu $n = 2k$ (n chẵn) thì trung vị mẫu là giá trị $\dfrac{x_k + x_{k+1}}{2}$

Trung vị không bị ảnh hưởng bởi các điểm ngoại lai (outliers).

\textbf{Tứ phân vị (Quartiles):} Giá trị trung vị chia mẫu dữ liệu đã sắp thứ tự thành 2 tập có số phần tử bằng nhau. Trung vị của tập dữ liệu nhỏ hơn là $Q_1$ (gọi là tứ phân vị dưới) và trung vị của tập dữ liệu lớn hơn là $Q_3$ (gọi là tứ phân vị trên). $Q_2$ được lấy bằng giá trị trung vị. Độ trải giữa, hay là khoảng tứ phân vị $IQR = Q_3 - Q_1$.

\textbf{Điểm Outlier:} còn gọi là điểm dị biệt, điểm ngoại lệ, điểm ngoại lai.... Đó là các phần tử của mẫu có giá trị nằm ngoài khoảng
\[ (Q_1 - 1.5 \times IQR; Q_3 + 1.5 \times IQR) \]

\subsection{Ước lượng}

Lý thuyết ước lượng là một nội dung trọng tâm trong thống kê và nghiên cứu khoa học, tập trung vào việc xác định giá trị của các tham số (parameters) của quần thể dựa trên những mẫu (samples) được chọn ra từ quần thể đó. Mục tiêu chính của ước lượng là tìm ra các giá trị gần đúng cho những đại lượng đặc trưng của quần thể như trung bình tổng thể ($\mu$), phương sai tổng thể ($\sigma^2$), và tỷ lệ phần tử có đặc điểm nhất định trong quần thể ($p$).

\begin{itemize}
    \item \textbf{Khoảng Tin Cậy (Confidence Interval - CI):} Là một loại ước lượng khoảng được sử dụng để chỉ ra phạm vi mà ta tin rằng tham số của tổng thể nằm trong đó. Khoảng tin cậy thường được xác định bởi hai giới hạn: giới hạn dưới và giới hạn trên. Ví dụ, một khoảng tin cậy 95\% cho trung bình tổng thể có thể là (20, 30), nghĩa là ta tin rằng với độ tin cậy 95\%, trung bình thực sự của tổng thể nằm trong khoảng từ 20 đến 30.

    \item \textbf{Mức Ý Nghĩa (Significance Level - $\alpha$):} Là ngưỡng mà ta chọn để quyết định ý nghĩa. Ví dụ mức ý nghĩa thường được chọn ở mức 0.05 - nghĩa là khả năng kết quả quan sát sự khác biệt được nhìn thấy trên số liệu là ngẫu nhiên chỉ là 5\%.

    \item \textbf{Độ Tin Cậy (Confidence Level):} Được biểu thị dưới dạng một tỷ lệ phần trăm chỉ mức độ tin tưởng hoặc sự chắc chắn mà khoảng tin cậy ước lượng của chúng ta bao gồm tham số tổng thể thực sự. Ví dụ: nếu ta xây dựng khoảng tin cậy với mức tin cậy 95\%, ta tin chắc rằng 95 trên 100 lần ước tính sẽ nằm giữa giá trị trên và giá trị dưới được chỉ định bởi khoảng tin cậy.
          \[ \gamma = 1 - \alpha \]
\end{itemize}


Có hai phương pháp ước lượng thường được sử dụng là ước lượng điểm (point estimation) và ước lượng khoảng (interval estimation), tuy nhiên trong phạm vi bài này, nhóm chỉ nhắc đến ước lượng bằng khoảng tin cậy.

% \begin{itemize}
%     \item \textbf{Ước lượng điểm (Point Estimation):} là dùng một tham số thống kê mẫu đơn lẻ để ước lượng giá trị tham số của tổng thể.
%           % Dùng align* để căn lề đẹp
%           \begin{align*}
%               \mu      & \approx \bar{x} \\
%               \sigma^2 & \approx s^2     \\
%               p        & \approx f
%           \end{align*}

%     \item \textbf{Ước lượng bằng khoảng tin cậy (Interval Estimation):} -Ước lượng bằng khoảng tin cậy chính là tìm ra khoảng ước lượng $(G_1; G_2)$ cho tham số $\theta$ trong tổng thể sao cho ứng với độ tin cậy (confidence) bằng $\gamma$ cho trước, $P(G_1 < \theta < G_2) = \gamma$.
% \end{itemize}

% \subsubsection{Ước lượng điểm (Point Estimation)}

% Một \textbf{ước lượng (estimator)} của một tham số (của tổng thể): là một biến ngẫu nhiên có giá trị phụ thuộc vào thông tin của mẫu, giá trị của nó là một xấp xỉ cho tham số chưa biết của tổng thể. Một giá trị cụ thể của biến ngẫu nhiên này gọi là một \textbf{giá trị ước lượng điểm}.

% Xét đại lượng ngẫu nhiên X có phân phối $F(x; \theta)$ với tham số $\theta$ chưa biết.

% Chọn một mẫu ngẫu nhiên cỡ n từ $X_1, X_2, \dots, X_n$.

% Thống kê $\hat{\theta} = h(X_1, X_2, \dots, X_n)$ gọi là một ước lượng điểm cho $\theta$.

% Với một mẫu cụ thể $(x_1, x_2, \dots, x_n)$, ta gọi $\hat{\theta} = h(x_1, x_2, \dots, x_n)$ là một giá trị ước lượng điểm cụ thể cho $\theta$.

% \subsubsection{Ước lượng bằng khoảng tin cậy - Bài toàn 1 mẫu}

% Cho tham số $\theta$ của tổng thể và $X_1, X_2, \dots, X_n$ là các quan sát ngẫu nhiên. Ta gọi khoảng $(c, d)$ là khoảng ước lượng (hay khoảng tin cậy) của tham số $\theta$ với độ tin cậy $\gamma$ nếu: $P(c < \theta < d) = \gamma$. Có thể nói, độ tin cậy $\gamma$ cho khoảng ước lượng của tham số $\theta$ chính là xác suất để ta đúng khi ước lượng tham số $\theta$ bằng khoảng $(c, d)$. Ngược lại, xác suất mà ta cho phép sai khi ước lượng $\theta$ được gọi là mức ý nghĩa. Kí hiệu là $\alpha$. Ta có $\alpha + \gamma = 1$.

% Xác định khoảng ước lượng đối xứng của trung bình tổng thể dựa vào một mẫu đã cho, với kích thước là $n$, trung bình mẫu là $\bar{x}$ và phương sai mẫu là $s^2$ hoặc phương sai tổng thể là $\sigma^2$. Mục tiêu là xác định một khoảng ước lượng đối xứng xung quanh $\mu$ của mẫu với một mức độ tin cậy cụ thể.

% Để giải quyết vấn đề này, ta cần đi đến việc xác định epsilon ($\epsilon$) - sai số ước lượng, dựa trên các thông tin đã biết về mẫu. Khoảng tin cậy sẽ được biểu diễn bằng khoảng $\bar{x} \pm \epsilon$. Tùy thuộc vào giả định về phân phối của dữ liệu và các thông tin đã biết về phương sai, cách tính $\epsilon$ sẽ thay đổi như sau:

% % Ghi chú: Tôi đã dùng \displaystyle để các phân số hiển thị lớn giống trong hình
% \begin{table}[H]
%     \centering
%     \label{tab:phanphoi_xs}
%     \begin{tabular}{|l|l|l|l|l|}
%         \hline
%         \textbf{Dạng}                                                                                                       & \textbf{Giả định}                                        & \textbf{Loại}                                                   & \textbf{Ngưỡng sai số}                                                       & \textbf{Khoảng tin cậy}                                                      \\ \hline
%         \multirow{3}{*}{Tỷ lệ}                                                                                              & \multirow{3}{*}{$n \ge 30$}                              & Đối xứng                                                        & $\displaystyle \epsilon = z_{\alpha/2} \sqrt{\frac{f(1-f)}{n}}$              & $\displaystyle f - \epsilon < p < f + \epsilon$                              \\ \cline{3-5}
%                                                                                                                             &                                                          & Bên phải                                                        & $\displaystyle \epsilon = z_{\alpha} \sqrt{\frac{f(1-f)}{n}}$                & $\displaystyle 0 < p < f + z_{\alpha} \sqrt{\frac{f(1-f)}{n}}$               \\ \cline{3-5}
%                                                                                                                             &                                                          & Bên trái                                                        & $\displaystyle \epsilon = z_{\alpha} \sqrt{\frac{f(1-f)}{n}}$                & $\displaystyle f - z_{\alpha} \sqrt{\frac{f(1-f)}{n}} < p < 1$               \\ \hline
%         \multirow{6}{*}{Trung bình}                                                                                         & \multirow{3}{*}{\parbox{3.5cm}{$X \sim N(\mu, \sigma^2)$                                                                                                                                                                                                                                 \\ Đã biết $\sigma$}} & Đối xứng & $\displaystyle \epsilon = z_{\alpha/2} \frac{\sigma}{\sqrt{n}}$ & $\displaystyle \bar{x} - \epsilon < \mu < \bar{x} + \epsilon$ \\ \cline{3-5}
%                                                                                                                             &                                                          & Bên phải                                                        & $\displaystyle \epsilon = z_{\alpha} \frac{\sigma}{\sqrt{n}}$                & $\displaystyle -\infty < \mu < \bar{x} + z_{\alpha} \frac{\sigma}{\sqrt{n}}$ \\ \cline{3-5}
%                                                                                                                             &                                                          & Bên trái                                                        & $\displaystyle \epsilon = z_{\alpha} \frac{\sigma}{\sqrt{n}}$                & $\displaystyle \bar{x} - z_{\alpha} \frac{\sigma}{\sqrt{n}} < \mu < \infty$  \\ \cline{2-5}
%                                                                                                                             & \multirow{3}{*}{\parbox{3.5cm}{$X \sim N(\mu, \sigma^2)$                                                                                                                                                                                                                                 \\ Chưa biết $\sigma$}} & Đối xứng & $\displaystyle \epsilon = t_{\alpha/2; n-1} \frac{s}{\sqrt{n}}$ & $\displaystyle \bar{x} - \epsilon < \mu < \bar{x} + \epsilon$ \\ \cline{3-5}
%                                                                                                                             &                                                          & Bên phải                                                        & $\displaystyle \epsilon = t_{\alpha; n-1} \frac{s}{\sqrt{n}}$                & $\displaystyle -\infty < \mu < \bar{x} + t_{\alpha; n-1} \frac{s}{\sqrt{n}}$ \\ \cline{3-5}
%                                                                                                                             &                                                          & Bên trái                                                        & $\displaystyle \epsilon = t_{\alpha; n-1} \frac{s}{\sqrt{n}}$                & $\displaystyle \bar{x} - t_{\alpha; n-1} \frac{s}{\sqrt{n}} < \mu < \infty$  \\ \hline
%         \multicolumn{2}{|l|}{\parbox{5cm}{Phân phối tùy ý, mẫu lớn ($n \ge 30$). Nếu chưa biết $\sigma$ thì thay bằng $s$}} & Đối xứng                                                 & $\displaystyle \epsilon = z_{\alpha/2} \frac{\sigma}{\sqrt{n}}$ & $\displaystyle \bar{x} - \epsilon < \mu < \bar{x} + \epsilon$                                                                                               \\ \cline{3-5}
%         \multicolumn{2}{|l|}{}                                                                                              & Bên phải                                                 & $\displaystyle \epsilon = z_{\alpha} \frac{\sigma}{\sqrt{n}}$   & $\displaystyle -\infty < \mu < \bar{x} + z_{\alpha} \frac{\sigma}{\sqrt{n}}$                                                                                \\ \cline{3-5}
%         \multicolumn{2}{|l|}{}                                                                                              & Bên trái                                                 & $\displaystyle \epsilon = z_{\alpha} \frac{\sigma}{\sqrt{n}}$   & $\displaystyle \bar{x} - z_{\alpha} \frac{\sigma}{\sqrt{n}} < \mu < \infty$                                                                                 \\ \hline
%     \end{tabular}
%     \caption{Cách tính $\epsilon$ và khoảng tin cậy cho một vài dạng phổ biến}
% \end{table}

% \subsubsection{Ước lượng bằng khoảng tin cậy - Bài toàn 2 mẫu}



\subsection{Kiểm định giả thuyết thống kê}

\subsubsection{Khái niệm chung về kiểm định}

Trong thống kê, \textbf{kiểm định (hypothesis testing)} là quá trình đánh giá một giả thuyết về dữ liệu để xác định xem liệu có đủ bằng chứng để chấp nhận hay bác bỏ giả thuyết đó. Mục tiêu của kiểm định là đưa ra quyết định dựa trên dữ liệu mẫu có sẵn để rút ra những kết luận về tổng thể.

Quá trình kiểm định thường bắt đầu bằng việc xây dựng hai giả thuyết:

\begin{itemize}
    \item \textbf{Giả thuyết không (null hypothesis, ký hiệu $H_0$):} là giả thiết về yếu tố cần kiểm định của tổng thể ở trạng thái bình thường, không chịu tác động của các hiện tượng liên quan. Yếu tố trong $H_0$ phải được xác định cụ thể.

    \item \textbf{Giả thuyết thay thế - giả thuyết đối (alternative hypothesis, ký hiệu $H_1$):} là một mệnh đề mâu thuẫn với $H_0$, $H_1$ thể hiện xu hướng cần kiểm định.
\end{itemize}

\textbf{Miền Bác Bỏ (Rejection Region):} là miền số thực thỏa $P(G \in RR / H_0 \text{ đúng}) = \alpha$. $\alpha$ là một số khá bé, thường không quá 10\% và được gọi là mức ý nghĩa của kiểm định. Một ký hiệu khác của miền bác bỏ được dùng trong bài: $W_\alpha$

\textbf{Miền Chấp Nhận (Acceptance Region):} phần bù của miền bác bỏ trong R

\textbf{Tiêu chuẩn kiểm định:} là hàm thống kê $G = G(X_1, X_2, \dots, X_n, \theta_0)$, xây dựng trên mẫu ngẫu nhiên $W = (X_1, X_2, \dots, X_n)$ và tham số $\theta_0$ liên quan đến $H_0$ ; Điều kiện đặt ra với thống kê G là nếu $H_0$ đúng thì quy luật phân phối xác suất của G phải hoàn toàn xác định.

\subsubsection{Quy tắc kiểm định:}

Từ mẫu thực nghiệm, ta tính được một giá trị cụ thể của tiêu chuẩn kiểm định, gọi là \textbf{giá trị kiểm định thống kê:}
\[ g_{qs} = G(x_1, x_2, \dots, x_n, \theta_0) \]

Theo nguyên lý xác suất bé, biến cố $G \in RR$ có xác suất nhỏ nên với 1 mẫu thực nghiệm ngẫu nhiên, nó không thể xảy ra.

Do đó:
\begin{itemize}
    \item + Nếu $g_{qs} \in RR$ thì bác bỏ $H_0$, thừa nhận giả thiết $H_1$.

    \item + Nếu $g_{qs} \notin RR$: ta chưa đủ dữ liệu khẳng định $H_0$ sai. Vì vậy ta chưa thể chứng minh được $H_1$ đúng.
\end{itemize}

\subsubsection{Các sai lầm trong bài toán kiểm định}

Kết luận của một bài toán kiểm định có thể mắc các sai lầm sau:

\begin{itemize}
    \item \textbf{Sai lầm loại I:} Bác bỏ giả thiết $H_0$ trong khi $H_0$ đúng. Xác suất mắc phải sai lầm này nếu $H_0$ đúng chính bằng mức ý nghĩa $\alpha$. Nguyên nhân mắc phải sai lầm loại I thường có thể do kích thước mẫu quá nhỏ, có thể do phương pháp lấy mẫu ...

    \item \textbf{Sai lầm loại II:} Thừa nhận $H_0$ trong khi $H_0$ sai, tức là mặc dù thực tế $H_1$ đúng nhưng giá trị thực nghiệm $g_{qs}$ không thuộc RR.
\end{itemize}

\begin{center}
    % Ghi chú: Cần có gói \usepackage{multirow} để chạy \multicolumn
    \begin{tabular}{|l|l|l|}
        \hline
                            & \multicolumn{2}{c|}{\textbf{Tình huống}}                                           \\ \cline{2-3}
        \textbf{Quyết định} & \multicolumn{1}{c|}{\textbf{$H_0$ đúng}} & \multicolumn{1}{c|}{\textbf{$H_0$ sai}} \\ \hline
        Bác bỏ $H_0$        & Sai lầm loại I. Xác suất = $\alpha$      & Quyết định đúng                         \\ \hline
        Không bác bỏ $H_0$  & Quyết định đúng                          & Sai lầm loại II. Xác suất = $\beta$     \\ \hline
    \end{tabular}
\end{center}

\subsubsection{Các bước thực hiện kiểm định}

\begin{enumerate}
    \item Phát biểu giả thuyết và đối thuyết của bài toán.
    \item Tính giá trị thống kê kiểm định (tiêu chuẩn kiểm định) cho bài toán.
    \item Xác định miền bác bỏ tốt nhất cho bài toán.
    \item Đưa ra kết luận.
\end{enumerate}

\subsection{Các mô hình kiểm định được sử dụng trong báo cáo}
\subsubsection{Bài toán kiểm định trung bình 1 mẫu}
\begin{table}[H]
    \centering
    \label{tab:kiemdinh_1mau}
    \renewcommand{\arraystretch}{1.5} % Tăng chiều cao dòng cho dễ đọc

    % Dùng resizebox để bảng tự co vừa chiều rộng trang giấy
    \resizebox{\textwidth}{!}{
        \begin{tabular}{|p{1.5cm}|p{3.5cm}|c|c|p{5.5cm}|c|}
            \hline
            \textbf{Dạng bài} & \textbf{Phân bố của tổng thể}                                                                                                   & \textbf{Giả thiết $H_0$}                                              & \textbf{Giả thiết đối $H_1$} & \centering\textbf{Miền bác bỏ RR} (miền bác bỏ $H_0$ với mức ý nghĩa $\alpha$) & \textbf{\parbox{4cm}{\centering Hàm thống kê kiểm định \\ (Tiêu chuẩn kiểm định)}} \\
            \hline

            % === PHẦN 1: KIỂM ĐỊNH TỶ LỆ ===
            \multirow{3}{*}{\parbox{1.5cm}{\textbf{Kiểm định tỷ lệ 1 mẫu}}}
                              & \multirow{3}{*}{\parbox{3.5cm}{* X có phân phối Không - một.                                                                                                                                                                                                                                                                                                                     \\ * $n \ge 30$. \hfill \textbf{(1)}}}
                              & \multirow{3}{*}{$p = p_0$}
                              & $p \neq p_0$                                                                                                                    & $(-\infty; -z_{\alpha/2}) \cup (z_{\alpha/2}; +\infty)$
                              & \multirow{3}{*}{$\displaystyle Z_{qs} = \frac{f - p_0}{\sqrt{\frac{p_0(1-p_0)}{n}}} \approx N(0,1)$}                                                                                                                                                                                                                                                                             \\
            \cline{4-5}

                              &                                                                                                                                 &                                                                       & $p < p_0$                    & $(-\infty; -z_{\alpha})$                                                       &                                                        \\
            \cline{4-5}
                              &                                                                                                                                 &                                                                       & $p > p_0$                    & \multicolumn{1}{r|}{$(z_{\alpha}; +\infty)$}                                   &                                                        \\
            \hline

            % === PHẦN 2: KIỂM ĐỊNH TRUNG BÌNH ===
            \multirow{9}{*}{\parbox{1.5cm}{\textbf{Kiểm định trung bình 1 mẫu}}}

            % --- Trường hợp 2a ---
                              & \multirow{3}{*}{\parbox{3.5cm}{* X có phân phối chuẩn.                                                                                                                                                                                                                                                                                                                           \\ * \textbf{Đã biết $\sigma^2$}. \hfill \textbf{(2a)}}}
                              & \multirow{3}{*}{$\mu = \mu_0$}
                              & $\mu \neq \mu_0$                                                                                                                & $(-\infty; -z_{\alpha/2}) \cup (z_{\alpha/2}; +\infty)$
                              & \multirow{3}{*}{$\displaystyle Z_{qs} = \frac{\bar{X} - \mu_0}{\frac{\sigma}{\sqrt{n}}} \sim N(0,1)$}                                                                                                                                                                                                                                                                            \\
            \cline{4-5}

                              &                                                                                                                                 &                                                                       & $\mu < \mu_0$                & $(-\infty; -z_{\alpha})$                                                       &                                                        \\
            \cline{4-5}
                              &                                                                                                                                 &                                                                       & $\mu > \mu_0$                & \multicolumn{1}{r|}{$(z_{\alpha}; +\infty)$}                                   &                                                        \\
            \cline{2-6}

            % --- Trường hợp 2b ---
                              & \multirow{3}{*}{\parbox{3.5cm}{* X có phân phối chuẩn.                                                                                                                                                                                                                                                                                                                           \\ * \textbf{Chưa biết $\sigma^2$}. \hfill \textbf{(2b)}}}
                              & \multirow{3}{*}{$\mu = \mu_0$}
                              & $\mu \neq \mu_0$                                                                                                                & $(-\infty; -t_{\alpha/2; (n-1)}) \cup (t_{\alpha/2; (n-1)}; +\infty)$
                              & \multirow{3}{*}{$\displaystyle T_{qs} = \frac{\bar{X} - \mu_0}{\frac{s}{\sqrt{n}}} \sim T_{(n-1)}$}                                                                                                                                                                                                                                                                              \\
            \cline{4-5}

                              &                                                                                                                                 &                                                                       & $\mu < \mu_0$                & $(-\infty; -t_{\alpha; (n-1)})$                                                &                                                        \\
            \cline{4-5}
                              &                                                                                                                                 &                                                                       & $\mu > \mu_0$                & \multicolumn{1}{r|}{$(t_{\alpha; (n-1)}; +\infty)$}                            &                                                        \\
            \cline{2-6}

            % --- Trường hợp 2c ---
                              & \multirow{3}{*}{\parbox{3.5cm}{* X có phân phối tùy ý.                                                                                                                                                                                                                                                                                                                           \\ * $n \ge 30$. \hfill \textbf{(2c)} \\ \footnotesize X không có giả thiết PPC}}
                              & \multirow{3}{*}{$\mu = \mu_0$}
                              & $\mu \neq \mu_0$                                                                                                                & $(-\infty; -z_{\alpha/2}) \cup (z_{\alpha/2}; +\infty)$
                              & \multirow{3}{*}{\parbox{4cm}{\centering $\displaystyle Z_{qs} = \frac{\bar{X} - \mu_0}{\frac{\sigma}{\sqrt{n}}} \approx N(0,1)$                                                                                                                                                                                                                                                  \\ \footnotesize Nếu chưa biết $\sigma$ thì thay bởi $s$}} \\
            \cline{4-5}

                              &                                                                                                                                 &                                                                       & $\mu < \mu_0$                & $(-\infty; -z_{\alpha})$                                                       &                                                        \\
            \cline{4-5}
                              &                                                                                                                                 &                                                                       & $\mu > \mu_0$                & \multicolumn{1}{r|}{$(z_{\alpha}; +\infty)$}                                   &                                                        \\
            \hline
        \end{tabular}
    }
    \caption{Công thức của bài toán kiểm định tỷ lệ \& trung bình 1 mẫu}
\end{table}

\subsubsection{Bài toán kiểm định 2 mẫu}
\begin{table}[h!]
    \centering
    \renewcommand{\arraystretch}{2.0} % Tăng chiều cao dòng lên 2.0 cho thoáng

    % Dùng resizebox để bảng tự động co vừa khít trang giấy
    \resizebox{\textwidth}{!}{
        \begin{tabular}{|p{5.5cm}|c|c|c|c|}
            \hline
            \textbf{Phân bố của tổng thể} & \textbf{GT $H_0$}                                                                                                               & \textbf{GT $H_1$}                                                     & \textbf{Miền bác bỏ RR}              & \textbf{T/chuẩn kiểm định} \\
            \hline

            % === (4a) ===
            \multirow{3}{*}{\parbox{5.5cm}{* 2 mẫu độc lập                                                                                                                                                                                                                                                              \\ * $X_1, X_2$ có pp chuẩn. \\ * Đã biết $\sigma_1^2$ và $\sigma_2^2$ \hfill (4a)}}
                                          & \multirow{3}{*}{$\mu_1 = \mu_2$}
                                          & $\mu_1 \neq \mu_2$                                                                                                              & $(-\infty; -z_{\alpha/2}) \cup (z_{\alpha/2}; +\infty)$
                                          & \multirow{3}{*}{$\displaystyle Z_{qs} = \frac{\bar{X}_1 - \bar{X}_2}{\sqrt{\frac{\sigma_1^2}{n_1} + \frac{\sigma_2^2}{n_2}}}$}                                                                                                                                              \\
            \cline{3-4}
                                          &                                                                                                                                 & $\mu_1 < \mu_2$                                                       & $(-\infty; -z_{\alpha})$             &                            \\
            \cline{3-4}
                                          &                                                                                                                                 & $\mu_1 > \mu_2$                                                       & $(z_{\alpha}; +\infty)$              &                            \\
            \hline

            % === (4b) ===
            \multirow{3}{*}{\parbox{5.5cm}{* 2 mẫu độc lập                                                                                                                                                                                                                                                              \\ * $X_1, X_2$ có pp chuẩn \\ * Chưa biết $\sigma_1^2; \sigma_2^2$; $\sigma_1^2 = \sigma_2^2$ \hfill (4b)}}
                                          & \multirow{3}{*}{$\mu_1 = \mu_2$}
                                          & $\mu_1 \neq \mu_2$                                                                                                              & \parbox{4.5cm}{\centering $(-\infty; -t_{\alpha/2;(n_1+n_2-2)}) \cup$                                                                     \\ $(t_{\alpha/2;(n_1+n_2-2)}; +\infty)$}
                                          & \multirow{3}{*}{$\displaystyle T_{qs} = \frac{\bar{X}_1 - \bar{X}_2}{\sqrt{s_p^2 \left(\frac{1}{n_1} + \frac{1}{n_2}\right)}}$}                                                                                                                                             \\
            \cline{3-4}
                                          &                                                                                                                                 & $\mu_1 < \mu_2$                                                       & $(-\infty; -t_{\alpha;(n_1+n_2-2)})$ &                            \\
            \cline{3-4}
                                          &                                                                                                                                 & $\mu_1 > \mu_2$                                                       & $(t_{\alpha;(n_1+n_2-2)}; +\infty)$  &                            \\
            \hline

            % === (4c) ===
            \multirow{3}{*}{\parbox{5.5cm}{* 2 mẫu độc lập                                                                                                                                                                                                                                                              \\ * $X_1, X_2$ có pp chuẩn \\ * Chưa biết $\sigma_1^2, \sigma_2^2$; $\sigma_1^2 \neq \sigma_2^2$ \hfill (4c)}}
                                          & \multirow{3}{*}{$\mu_1 = \mu_2$}
                                          & $\mu_1 \neq \mu_2$                                                                                                              & \parbox{4.5cm}{\centering $(-\infty; -t_{\alpha/2;(\nu)}) \cup$                                                                           \\ $(t_{\alpha/2;(\nu)}; +\infty)$}
                                          & \multirow{3}{*}{$\displaystyle T_{qs} = \frac{\bar{X}_1 - \bar{X}_2}{\sqrt{\frac{s_1^2}{n_1} + \frac{s_2^2}{n_2}}}$}                                                                                                                                                        \\
            \cline{3-4}
                                          &                                                                                                                                 & $\mu_1 < \mu_2$                                                       & $(-\infty; -t_{\alpha;(\nu)})$       &                            \\
            \cline{3-4}
                                          &                                                                                                                                 & $\mu_1 > \mu_2$                                                       & $(t_{\alpha;(\nu)}; +\infty)$        &                            \\
            \hline

            % === (4d) ===
            \multirow{3}{*}{\parbox{5.5cm}{* 2 mẫu độc lập                                                                                                                                                                                                                                                              \\ * $X_1, X_2$ có pp tùy ý \\ * Mẫu lớn: $n_1, n_2 \ge 30$ \\ * Đã biết hoặc chưa biết $\sigma_1^2, \sigma_2^2$ \hfill (4d)}}
                                          & \multirow{3}{*}{$\mu_1 = \mu_2$}
                                          & $\mu_1 \neq \mu_2$                                                                                                              & $(-\infty; -z_{\alpha/2}) \cup (z_{\alpha/2}; +\infty)$
                                          & \multirow{3}{*}{$\displaystyle Z_{qs} = \frac{\bar{X}_1 - \bar{X}_2}{\sqrt{\frac{\sigma_1^2}{n_1} + \frac{\sigma_2^2}{n_2}}}$}                                                                                                                                              \\
            \cline{3-4}
                                          &                                                                                                                                 & $\mu_1 < \mu_2$                                                       & $(-\infty; -z_{\alpha})$             &                            \\
            \cline{3-4}
                                          &                                                                                                                                 & $\mu_1 > \mu_2$                                                       & $(z_{\alpha}; +\infty)$              &                            \\
            \hline

            % === (4e) ===
            \multirow{3}{*}{\parbox{5.5cm}{* 2 mẫu phụ thuộc                                                                                                                                                                                                                                                            \\ tương ứng theo cặp \\ * $X_1, X_2$ có pp chuẩn \\ * Chưa biết $\sigma_D^2$ \hfill (4e)}}
                                          & \multirow{3}{*}{$\mu_1 = \mu_2$}
                                          & $\mu_1 \neq \mu_2$                                                                                                              & \parbox{4.5cm}{\centering $(-\infty; -t_{\alpha/2;(n-1)}) \cup$                                                                           \\ $(t_{\alpha/2;(n-1)}; +\infty)$}
                                          & \multirow{3}{*}{$\displaystyle T_{qs} = \frac{\bar{X}_D - \mu_0}{s_D} \sqrt{n}$}                                                                                                                                                                                            \\
            \cline{3-4}
                                          &                                                                                                                                 & $\mu_1 < \mu_2$                                                       & $(-\infty; -t_{\alpha;(n-1)})$       &                            \\
            \cline{3-4}
                                          &                                                                                                                                 & $\mu_1 > \mu_2$                                                       & $(t_{\alpha;(n-1)}; +\infty)$        &                            \\
            \hline

            % === (4f) ===
            \multirow{3}{*}{\parbox{5.5cm}{* 2 mẫu phụ thuộc                                                                                                                                                                                                                                                            \\ tương ứng theo cặp \\ * 2 mẫu lớn: $n \ge 30$ \\ * Đã biết hoặc chưa biết $\sigma_D^2$ \hfill (4f)}}
                                          & \multirow{3}{*}{$\mu_1 = \mu_2$}
                                          & $\mu_1 \neq \mu_2$                                                                                                              & $(-\infty; -z_{\alpha/2}) \cup (z_{\alpha/2}; +\infty)$
                                          & \multirow{3}{*}{$\displaystyle Z_{qs} = \frac{\bar{X}_D - \mu_0}{\sigma_D} \sqrt{n}$}                                                                                                                                                                                       \\
            \cline{3-4}
                                          &                                                                                                                                 & $\mu_1 < \mu_2$                                                       & $(-\infty; -z_{\alpha})$             &                            \\
            \cline{3-4}
                                          &                                                                                                                                 & $\mu_1 > \mu_2$                                                       & $(z_{\alpha}; +\infty)$              &                            \\
            \hline
        \end{tabular}
    }
    \caption{Các dạng toán kiểm định 2 mẫu}
\end{table}

\subsubsection{Phân tích phương sai (anova)}

\textbf{Phân tích phương sai} là một mô hình dùng để xem xét sự biến động của một biến ngẫu nhiên định lượng X chịu tác động trực tiếp của một hay nhiều yếu tố nguyên nhân (định tính). Được làm hai loại là phân tích phương sai 1 yếu tố và phân tích phương sai 2 yếu tố.

\subsubsection{Phân tích phương sai 1 yếu tố}

\textbf{Giả thiết}
\begin{itemize}
    \item Các tổng thể có phân phối chuẩn $N(\mu_i; \sigma_i^2)$, $i = 1, 2, \dots, k$ với $k$ là tổng thể (thông thường $k \ge 3$).
    \item Phương sai các tổng thể chưa biết nhưng được giả định là bằng nhau $(\sigma_1^2 = \sigma_2^2 = \dots = \sigma_k^2)$.
    \item Các mẫu quan sát (từ k tổng thể) được lấy độc lập.
\end{itemize}

\textbf{Các bước thực hiện}

\textbf{Bước 1: Đặt giả thiết kiểm định}
\begin{align*}
    H_0 & : \mu_1 = \mu_2 = \dots = \mu_k,            \\
    H_1 & : \exists \mu_i \neq \mu_j \quad (i \neq j)
\end{align*}

\textbf{Bước 2:} Tính trung bình mẫu của các nhóm $\bar{x}_1, \bar{x}_2, \dots, \bar{x}_k$ theo công thức:
\[ \bar{x}_i = \frac{\sum_{j=1}^{n_i} x_{ij}}{n_i}, \quad (i = 1, 2, \dots, k) \]

\textbf{Bước 3:} Tính tổng các bình phương lệch (tổng bình phương):
\begin{align*}
    SSW & = \sum_{i=1}^{k} \sum_{j=1}^{n_i} (x_{ij} - \bar{x}_i)^2           \\
    SSB & = \sum_{i=1}^{k} n_i (\bar{x}_i - \bar{x})^2                       \\
    SST & = SSW + SSB = \sum_{i=1}^{k} \sum_{j=1}^{n_i} (x_{ij} - \bar{x})^2
\end{align*}

\textbf{Bước 4:} Tính các phương sai:
\[ MSW = \frac{SSW}{N - k}, \quad MSB = \frac{SSB}{k - 1} \]

\textbf{Bước 5:} Tính thống kê kiểm định (tiêu chuẩn kiểm định, giá trị quan sát):
\[ F = \frac{MSB}{MSW} \]

\textbf{Bước 6:} Xác định miền bác bỏ của bài toán:
\[ RR = (F_{\alpha; k-1; N-k}; +\infty) \]
Tìm giá trị $F_{\alpha; k-1; N-k}$ tra bảng Fisher mức ý nghĩa $\alpha$ và cột $k - 1$ và dòng $N - k$.

\textbf{Bước 7:} Đưa ra kết luận:
\begin{itemize}
    \item Nếu $F > F_{\alpha; k-1; N-k} \iff F \in RR \Rightarrow$ Bác bỏ $H_0$, chấp nhận $H_1$
    \item Nếu $F < F_{\alpha; k-1; N-k} \iff F \notin RR \Rightarrow$ không bác bỏ $H_0$ (chưa bác bỏ được $H_0$, chấp nhận $H_0$)
\end{itemize}

% Bảng tóm tắt ANOVA
\begin{table}[h]
    \centering
    \caption{Bảng tóm tắt ANOVA 1 yếu tố}
    \begin{tabular}{|l|c|c|c|c|}
        \hline
        \textbf{Nguồn của sự biến thiên} & \textbf{SS} & \textbf{df} & \textbf{MS} & \textbf{F}                                           \\ \hline
        Giữa các nhóm                    & SSB         & k-1         & MSB         & \multirow{2}{*}{$\displaystyle F = \frac{MSB}{MSW}$} \\ \cline{1-4}
        Trong từng nhóm                  & SSW         & N-k         & MSW         &                                                      \\ \hline
        Toàn bộ                          & SST         & N-1         &             &                                                      \\ \hline
    \end{tabular}
\end{table}


\subsubsection{Hồi quy tuyến tính bội}
\textbf{Khái niệm:} Hồi quy tuyến tính là một kỹ thuật phân tích dữ liệu dự đoán giá trị của dữ liệu không xác định bằng cách sử dụng một giá trị dữ liệu liên quan và đã biết khác.

Bài toán phân tích hồi quy là bài toán nghiên cứu mối liên hệ phụ thuộc của một biến (gọi là biến phụ thuộc) vào một hay nhiều biến khác (gọi là các biến độc lập), với ý tưởng ước lượng được giá trị trung bình (tổng thể) của biến phụ thuộc theo giá trị của các biến độc lập, dựa trên mẫu được biết trước.

Trong hồi quy tuyến tính đơn, chỉ có một biến độc lập được sử dụng để dự đoán biến phụ thuộc. Tuy nhiên, trong hồi quy tuyến tính bội, có nhiều hơn một biến độc lập được sử dụng.

Hàm hồi quy tuyến tính đơn có dạng:
\[ Y = f_Y(X) = E(Y|X) = \beta_0 + \beta_1 X. \]

Hàm hồi quy tuyến tính bội có dạng:
\[ Y = f_Y(X_1; X_2; \dots; X_k) = E(Y | (X_1; X_2; \dots; X_k)) = \beta_0 + \beta_1 X_1 + \beta_2 X_2 + \dots + \beta_k X_k. \]
Trong đó:
\begin{itemize}
    \item $Y$ là biến phụ thuộc (biến cần dự đoán).
    \item $X_1, X_2, \dots, X_k$ là các biến độc lập (biến giải thích).
    \item $\beta_0$ (cũng được gọi là hệ số điều chỉnh) là hệ số mức độ tự do của mô hình, tức là giá trị dự đoán của biến phụ thuộc khi tất cả các biến độc lập đều bằng 0.
    \item $\beta_1, \beta_2, \dots, \beta_k$ là các hệ số hồi quy tương ứng với từng biến độc lập.
    \item $\epsilon$ là sai số ngẫu nhiên (error term) biểu thị sự khác biệt giữa giá trị thực tế và giá trị dự đoán bởi mô hình.
\end{itemize}
\newpage
\section{Tiền xử lý dữ liệu}
\subsection{Đọc dữ liệu vào R}

Ta đọc dữ liệu từ file dữ liệu đã cho dưới định dạng CSV vào R bằng hàm \verb|read.csv()| như sau:

\begin{lstlisting}[language=R, caption={Đọc dữ liệu từ file trong R}]
df <- read.csv("./data_sets/All_GPUs.csv")
head(df, 5)
\end{lstlisting}


Sau khi đọc dữ liệu xong, ta có thể sử dụng hàm \verb|head()| để hiển thị 5 dòng đầu tiên của dữ liệu nhằm kiểm tra xem dữ liệu đã được đọc đúng chưa (Hình \ref{fig:head_data}).

\begin{figure}[H]
  \centering
  \includegraphics[width=0.8\textwidth]{graphics/2_1_data.jpg}
  \caption{Hiển thị 5 dòng đầu tiên của dữ liệu sau khi đọc vào R}
  \label{fig:head_data}
\end{figure}


Như vậy, ta đã hoàn thành việc đọc dữ liệu từ file CSV vào R và có thể tiến hành các bước tiền xử lý dữ liệu tiếp theo.
\newpage
\subsection{Làm sạch dữ liệu}


Vì trong file dữ liệu ban đầu, có thể tồn tại các giá trị bị thiếu (NA) hoặc các giá trị không hợp lệ, ta cần thực hiện các bước làm sạch dữ liệu để đảm bảo tính chính xác và độ tin cậy của phân tích sau này.


Trước tiên, ta sẽ phải thay thế tất cả các giá trị rác, không hợp lệ thành giá trị $NA$ trong R. Ví dụ, nếu một ô bất kì có giá trị là chuỗi rỗng \verb|""| hoặc ký tự đặc biệt như \verb|"N/A"|, ta sẽ thay thế chúng bằng $NA$ như sau:

\begin{lstlisting}[language=R, caption={Thay thế giá trị rác thành NA}, captionpos=b]
df <- df %>%
  mutate(across(where(is.character), trimws))
df[df == ""] <- NA
df[df == "N/A"] <- NA
df[df == "NA"] <- NA
df[df == "-"] <- NA
df[df == "Unknown Release Date"] <- NA
# Chi lay nam san xuat, khong lay ngay cu the
df$Release_Date <- as.Date(df$Release_Date, format = "%d-%b-%Y")
df$Release_Date <- format(df$Release_Date, "%Y")
\end{lstlisting}


Sau khi thay thế các giá trị rác, nhóm nhận thấy rằng có rất nhiều yếu tố có số lượng $NA$ lớn, điều này buộc nhóm phải lựa chọn giữa loại bỏ và chuẩn hoá. Trong nội dung bài báo cáo này, nhóm sẽ loại bỏ những đăc điểm (cột) có số lượng giá trị $NA$ vượt quá 15\% tổng số dòng dữ liệu. Để thực hiện việc này, ta có thể sử dụng đoạn mã sau:

\begin{lstlisting}[language=R, caption={Trực quan hoá tỷ lệ dữ liệu khuyết thiếu}, captionpos=b]
# Dem so luong gia tri NA trong moi cot
missing_counts = freq.na(df)
# Ve do thi ty le du lieu khuyet
ggplot(missing_counts, aes(x = rownames(missing_counts), y = missing_counts[,2], )) +
  geom_bar(stat = "identity", fill = "cyan") +
  geom_text(aes(label = paste0(missing_counts[,2], "%")), vjust = -0.5, size = 2) +
  labs(title = "Missing rate", x = "Feature", y = "Rate (%)") +
  theme_minimal() +
  theme(axis.text.x = element_text(
    size = 10,
    angle = 90,
    hjust = 1
  ))
\end{lstlisting}


Kết quả trực quan hóa tỷ lệ dữ liệu khuyết thiếu được thể hiện trong Hình \ref{fig:missing_rate}.

\begin{figure}[H]
  \centering
  \includegraphics[width=0.9\textwidth]{graphics/2_2_missing_rate.jpg}
  \caption{Tỷ lệ dữ liệu khuyết thiếu trong các đặc trưng}
  \label{fig:missing_rate}
\end{figure}


Tiếp theo sau đó, ta sẽ loại bỏ các cột có tỷ lệ giá trị $NA$ vượt quá 15\% tổng số dòng dữ liệu như sau:

\begin{lstlisting}[language=R, caption={Loại bỏ các cột có tỷ lệ NA > 15\%}, captionpos=b]
missing_counts_df <- data.frame(
  feature = rownames(missing_counts),
  percent = missing_counts[,2]
)
cols_to_keep <- missing_counts_df$feature[missing_counts_df$percent <= 15 & missing_counts_df$feature != "Architecture" & missing_counts_df$feature != "Name"]
df_filtered <- df[, cols_to_keep, drop = FALSE]
head(df_filtered, 5)
df_filtered <- na.omit(df_filtered)
\end{lstlisting}


Bằng câu lệnh \verb|print(names(df_filtered))|, ta có thể kiểm tra lại các cột còn lại sau khi đã loại bỏ các cột có tỷ lệ giá trị $NA$ vượt quá 15\% (Hình \ref{fig:final_columns}).
\begin{figure}[H]
  \centering
  \includegraphics[width=0.8\textwidth]{graphics/2_2_final_columns.jpg}
  \caption{Các cột còn lại sau khi loại bỏ các cột có tỷ lệ NA > 15\%}
  \label{fig:final_columns}
\end{figure}


Mặc dù đã làm sạch dữ liệu bằng cách loại bỏ các cột có tỷ lệ khuyết hoặc số lượng giá trị $NA$ cao, vẫn còn một yếu tố khiến việc phân tích dữ liệu trở nên khó khăn, đó là các đơn vị đo, do đó ta cần chuẩn hoá các đơn vị đo trong dữ liệu bằng cách loại bỏ chúng.

\begin{lstlisting}[language=R, caption={Chuẩn hoá các đơn vị đo trong dữ liệu}, captionpos=b]
# Chuan hoa don vi do trong cac cot
remove_unit_cols <- c("Memory_Bandwidth", "Memory_Speed", "Memory_Bus", "Direct_X")
main_df <- df_filtered
main_df[remove_unit_cols] <- lapply(df_filtered[remove_unit_cols], function(x) {
  as.numeric(gsub("[^0-9.]", "", x))
})
clean_cache <- function(x) {
  main <- as.numeric(sub("KB.*", "", x))      # 2304
  mult      <- as.numeric(sub(".*\\(x([0-9]+)\\)", "\\1", x))  # 2
  if (is.na(mult)) mult <- 1
  return(main * mult)
}
main_df$L2_Cache <- sapply(main_df$L2_Cache, clean_cache)
\end{lstlisting}

Dữ liệu đã được làm sạch được lưu vào biến \verb|main_df| và được hiện trong Hình \ref{fig:cleaned_data}.
\begin{figure}[H]
  \centering
  \includegraphics[width=0.6\textwidth]{graphics/2_2data_cleaned.png}
  \caption{Dữ liệu sau khi làm sạch}
  \label{fig:cleaned_data}
\end{figure}
\newpage
\section{Thống kê mô tả}

Sau khi đã loại bỏ các các biến có tỉ lệ dữ liệu khuyết cao, nhóm tiến hành chọn lọc các biến quan trọng. Sau quá trình tham khảo, nhóm nhận thấy thông số quyết định hiệu suất xử lí của GPU là băng thông bộ nhớ (Memory\_Bandwidth) vì nó quyết định tốc độ truyền tải dữ liệu giữa bộ nhớ và GPU, đặc biệt với các tác vụ yêu cầu GPU xử lí lượng dữ liệu lớn. Băng thông càng cao thì tốc độ xử lý và khả năng đa nhiệm của hệ thống càng nhanh và ổn định. Ngoài ra, nhóm lựa chọn các biến có tầm ảnh hưởng khác nhau đến Memory\_Bandwidth gồm: Manufacturer, Process, Memory\_Speed, Memory\_Bus, Memory\_Type, L2\_Cache, Dedicated.

\subsection{Tính các giá trị đặc trưng của các biến.}

\subsubsection{Đối với các biến có giá trị số.}
Ta tiến hành lấy các biến trong dữ liệu mẫu ra để tính các giá trị đặc trưng của các biến như sau:

\begin{lstlisting}[language=R, caption=Tính các giá trị đặc trưng của các biến]
numeric_data <- main_df[, c("Process", "Memory_Speed", "Memory_Bus" , "L2_Cache")]
summary(numeric_data)
\end{lstlisting}

\underline{\textbf{Kết quả:}}

\begin{figure}[H]
    \centering
    \includegraphics[width=0.8\textwidth]{graphics/4_1tkmt.png}
    \caption{Kết quả tính các giá trị đặc trưng của các biến}
    \label{fig:4_1tkmt}
\end{figure}

\underline{\textbf{Nhận xét:}} từ kết quả ở hình \ref{fig:4_1tkmt}, ta có thể xem giá trị lớn nhất, nhỏ nhất của các biến cũng như giá trị trung bình của chúng.

Nhóm cũng thử vẽ biểu đồ hộp để trực quan hóa trực quan hoá các số liệu trong hình \ref{fig:4_1tkmt}:
\begin{lstlisting}[language=R, caption=Vẽ biểu đồ hộp cho các biến số]
par(mfrow = c(2, 2)) # Chia khung thanh 2 hang va 2 cot

boxplot(main_df$Process, main = "Process (nm)", col = "lightblue", ylab = "Value")
boxplot(main_df$Memory_Speed, main = "Memory Speed (MHz)", col = "lightgreen", ylab = "Value")
boxplot(main_df$Memory_Bus, main = "Memory Bus (Bit)", col = "pink", ylab = "Value")
boxplot(main_df$L2_Cache, main = "L2 Cache (KB)", col = "wheat", ylab = "Value")

par(mfrow = c(1, 1)) # Chinh lai nhu ban dau

\end{lstlisting}

\underline{\textbf{Kết quả:}}

\begin{figure}[H]
    \centering
    \includegraphics[width=0.8\textwidth]{graphics/4_1box.png}
    \caption{Biểu đồ hộp cho các biến số}
    \label{fig:4_1boxplot}
\end{figure}

\underline{\textbf{Nhận xét:}} từ biểu đồ hộp ở hình \ref{fig:4_1boxplot}, ta thấy rằng các biến có một vài ngoại lai nằm ngoài phạm vi của biểu đồ hộp. Bên cạnh đó biểu đồ hợp cũng cho thấy sự khớp như kết quả ở phần trên.


\subsubsection{Đối với các biến phân loại}
Đối với các biến phân loại như Manufacturer, Memory\_Type, Dedicated, ta có thể đếm số lượng các nhóm con trong các biến này và tần số xuất hiện của chúng như sau:

\begin{lstlisting}[language=R, caption=Đếm số lượng các nhóm con trong các biến phân loại]
table(main_df$Manufacturer)
table(main_df$Memory_Type)
table(main_df$Dedicated)
\end{lstlisting}


\underline{\textbf{Kết quả:}}

\begin{figure}[H]
    \centering
    \includegraphics[width=0.8\textwidth]{graphics/4_1dem.png}
    \caption{Kết quả đếm số lượng các nhóm con trong các biến phân loại}
    \label{fig:4_2tkmt}
\end{figure}

\underline{\textbf{Nhận xét:}} từ kết quả ở hình \ref{fig:4_2tkmt}, ta có thể thấy được rằng có 4 hãng sản xuất GPU chính trong dữ liệu là Nvidia, AMD, Intel và ATI. Trong đó Nvidia và AMD chiếm đa số, còn Intel và ATI chỉ chiếm một phần rất nhỏ. Ngoài ra, ta có 11 loại bộ nhớ khác nhau. Về Dedicated ta chỉ có 2 giá trị là Yes và No.


\subsection{Sự phân phối tần số của biến Memory\_Bandwidth.}

Sự phân phối tần số của biến Memory\_Bandwidth được thể hiện qua biểu đồ tần số sau:

\begin{figure}[H]
    \centering
    \includegraphics[width=0.8\textwidth]{graphics/4_1bwhist.png}
    \caption{Biểu đồ tần số của biến Memory\_Bandwidth}
    \label{fig:4_3hist}
\end{figure}

\underline{\textbf{Nhận xét:}} từ biểu đồ tần số ở hình \ref{fig:4_3hist}, ta thấy rằng sự phân phối của biến Memory\_Bandwidth của các GPU trong tập dữ liệu. Biến không tuân theo phân phối chuẩn, phân phối lệch phải, biến có phần lớn giá trị tập trung thấp, chủ yếu ở ngưỡng dưới 200 GB/s và có xu hướng giảm dần khi giá trị của “Memory\_Bandwidth” tăng lên. Điều này phản ánh nhu cầu thị trường đối với các loại GPU. Các trường hợp có băng thông bộ nhớ cao là rất hiếm, có sự chênh lệch lớn giữa các đối tượng quan sát được, cho thấy một số loại GPU hiệu năng cao có khả năng được sử dụng cho các công việc đặc thù.

% biến Memory\_Bandwidth có phân phối lệch phải, với phần lớn các giá trị tập trung ở bên trái biểu đồ. Điều này cho ta thấy rằng biến Memory\_Bandwidth không tuân theo phân phối chuẩn.


\subsection{Đồ thị scatter plot cho biến Memory\_Bandwidth theo Memory\_Bus, L2\_Cache, Memory\_Speed, Process.}

Ta tiến hành vẽ các đồ thị scatter plot để quan sát mối quan hệ giữa biến Memory\_Bandwidth với các biến Memory\_Bus, L2\_Cache, Memory\_Speed, Process như sau:
\begin{lstlisting}[language=R, caption=Vẽ đồ thị scatter plot.]
ve_bieu_do_tuy_chinh <- function(du_lieu_x, du_lieu_y, nhan_x, nhan_y) {
  plot(x = du_lieu_x, 
       y = du_lieu_y,
       main = paste("Moi quan he giua", nhan_x, "va", nhan_y),
       xlab = nhan_x,
       ylab = nhan_y,
       pch = 19,          # Style
       col = "darkblue",  # Color
       frame = FALSE)     # Remove extra frame
}
\end{lstlisting}

\underline{\textbf{Kết quả:}}

\begin{figure}[H]
    \centering
    \includegraphics[width=0.8\textwidth]{graphics/4_1scatter.png}
    \caption{Đồ thị scatter plot cho biến Memory\_Bandwidth theo Memory\_Bus, L2\_Cache, Memory\_Speed, Process}
    \label{fig:4_4scatter}
\end{figure}

\underline{\textbf{Nhận xét:}}

\begin{itemize}
    \item Dựa vào đồ thị scatter plot cho Memory\_Bandwidth theo Memory\_Bus, nhóm thấy được rằng phần lớn điểm dữ liệu tập trung ở vùng Memory\_Bus nhỏ dưới 1000 bit, với băng thông không quá 500 GB/s. Điều này cho thấy đây là là các Bus bộ nhớ phổ biến cho các hệ thống bộ nhớ. Mối quan hệ phụ thuộc giữa Memory Bandwidth và Memory Bus chỉ ở mức trung bình. Có sự phân tán rất lớn trong băng thông bộ nhớ, cho thấy rằng ngoài Memory Bus còn nhiều yếu tố khác ảnh hưởng đến Memory\_Bandwidth.
    \item Dựa vào đồ thị scatter plot cho Memory\_Bandwidth theo L2\_Cache, nhóm thấy được rằng phần lớn các điểm dữ liệu tập trung ở vùng Cache nhỏ, dưới 2000 KB, với băng thông bộ nhớ từ thấp đến trung bình. Từ đây cho thấy đa số hệ thống trong tập dữ liệu có dung lượng L2\_Cache khiêm tốn. Ở vùng L2 Cache lớn hơn, trên mức 4000 KB, băng thông bộ nhớ có độ phân tán rất rộng, từ mức trung bình đến cao.
    \item Dựa vào đồ thị scatter plot cho Memory\_Bandwidth theo Memory\_Speed, nhóm thấy được rằng Phần lớn dữ liệu tập trung ở vùng bộ nhớ thấp, dưới 1000 MHz, băng thông trải dài từ thấp đến trung bình. Giữa Memory\_Bandwidth và Memory\_Speed có mối quan hệ phụ thuộc rõ ràng, dù ở băng thông cao hơn vẫn có sự phân tán dữ liệu khá lớn. Khi tốc độ bộ nhớ tăng lên thì băng thông bộ nhớ cũng có xu hướng tăng dần, điều này cho thấy Memory\_Speed có ảnh hưởng mạnh mẽ đến Memory\_Bandwidth.


\end{itemize}
\newpage
\section{Thống kê suy diễn}
\subsection{Phân tích ANOVA hai yếu tố về ảnh hưởng của Hãng sản xuất và Loại bộ nhớ lên Băng thông bộ nhớ}

\subsubsection{Mục đích kiểm định}

Mục đích của kiểm định này là đánh giá xem:

\begin{enumerate}
    \item Băng thông bộ nhớ trung bình có khác nhau giữa các \textbf{hãng} hay không.
    \item Băng thông bộ nhớ trung bình có khác nhau giữa các \textbf{loại GPU} (tích hợp so với chuyên dụng) hay không.
    \item Có tồn tại \textbf{tác động tương tác} giữa Hãng và Loại bộ nhớ đối với băng thông bộ nhớ hay không.
\end{enumerate}
\subsubsection{Giả thiết nghiên cứu}

\begin{enumerate}
    \item \textbf{Ảnh hưởng của hãng}
          \begin{itemize}
              \item \textbf{Giả thiết không $H_0$}: Băng thông bộ nhớ trung bình của các hãng là như nhau.
              \item \textbf{Giả thiết đối $H_1$}: Có ít nhất hai hãng có băng thông bộ nhớ trung bình khác nhau.
          \end{itemize}
    \item \textbf{Ảnh hưởng của loại GPU}
          \begin{itemize}
              \item \textbf{Giả thiết không $H_0$}: Băng thông bộ nhớ trung bình giữa các loại GPU là như nhau.
              \item \textbf{Giả thiết đối $H_1$}: Có sự khác biệt băng thông trung bình giữa ít nhất hai loại GPU.
          \end{itemize}
    \item \textbf{Tác động tương tác giữa hai yếu tố}
          \begin{itemize}
              \item \textbf{Giả thiết không $H_0$}: Không tồn tại tương tác giữa hãng và loại GPU (tác động của một yếu tố không phụ thuộc vào yếu tố kia).
              \item \textbf{Giả thiết đối $H_1$}: Có tương tác giữa hãng và loại GPU (tác động của một yếu tố phụ thuộc vào yếu tố còn lại).
          \end{itemize}
\end{enumerate}
% \textbf{1. Ảnh hưởng của hãng}
% \textbf{2. Ảnh hưởng của loại GPU}
% \textbf{3. Tác động tương tác giữa hai yếu tố}

% ==========================================================
% \subsubsection{Các giả định của ANOVA hai yếu tố}
% ANOVA hai yếu tố yêu cầu ba giả định chính:

% \begin{enumerate}
%     \item \textbf{Phần dư tuân theo phân phối chuẩn.}
%           \begin{itemize}
%               \item \textbf{Ý nghĩa}: Nếu phần dư phân phối chuẩn, mô hình ANOVA sẽ cho kết quả đáng tin cậy.
%               \item Kiểm tra bằng \textbf{Shapiro–Wilk test}.
%               \item \textbf{Giả thiết kiểm định}:
%                     \begin{itemize}
%                         \item $H_0$: phần dư có phân phối chuẩn.
%                         \item $H_1$: phần dư không có phân phối chuẩn.
%                     \end{itemize}
%           \end{itemize}
%     \item \textbf{Phương sai giữa các nhóm là đồng nhất.}
%           \begin{itemize}
%               \item \textbf{Ý nghĩa}: Các nhóm có cùng độ biến thiên thì phép so sánh trung bình mới hợp lệ.
%               \item Kiểm tra bằng \textbf{Levene’s Test}.
%               \item \textbf{Giả thiết kiểm định}:
%                     \begin{itemize}
%                         \item $H_0$: các nhóm có phương sai bằng nhau.
%                         \item $H_1$: có ít nhất hai nhóm có phương sai khác nhau.
%                     \end{itemize}
%           \end{itemize}
%     \item \textbf{Các quan sát độc lập.}
%           \begin{itemize}
%               \item \textbf{Ý nghĩa}: mỗi GPU trong dữ liệu được coi là bằng chứng độc lập.
%               \item Đây là giả định về thiết kế dữ liệu (không kiểm định bằng thống kê).
%           \end{itemize}
% \end{enumerate}

% \textbf{1. Phần dư tuân theo phân phối chuẩn}
% \textbf{2. Phương sai giữa các nhóm là đồng nhất}
% \textbf{3. Các quan sát độc lập}


% ==========================================================
% \subsubsection{Cơ sở toán học của ANOVA hai yếu tố}



% \textbf{Mô hình ANOVA hai yếu tố}

% Giả sử ta quan sát giá trị Băng thông bộ nhớ được ký hiệu là $Y_{ijk}$, trong đó:
% \begin{itemize}
%     \item $i$ là chỉ số hãng
%     \item $j$ là chỉ số loại GPU
%     \item $k$ là chỉ số mẫu trong từng nhóm
% \end{itemize}

% Mô hình ANOVA hai yếu tố được viết dưới dạng:
% \[
%     Y_{ijk} = \mu + \alpha_i + \beta_j + (\alpha\beta)_{ij} + \varepsilon_{ijk}
% \]

% Trong đó:
% \begin{itemize}
%     \item $\mu$: trung bình chung của toàn bộ dữ liệu
%     \item $\alpha_i$: ảnh hưởng của hãng thứ $i$
%     \item $\beta_j$: ảnh hưởng của loại GPU (chuyên dụng và tích hợp)
%     \item $(\alpha\beta)_{ij}$: ảnh hưởng tương tác giữa hãng và loại GPU
%     \item $\varepsilon_{ijk}$: sai số ngẫu nhiên, tuân theo phân phối chuẩn
% \end{itemize}

% \textbf{Ý nghĩa của các thành phần trong mô hình}

% \begin{itemize}
%     \item $\alpha_i$ đo lường mức độ hãng làm thay đổi băng thông bộ nhớ so với trung bình chung.
%     \item $\beta_j$ đo lường mức độ loại GPU làm thay đổi băng thông bộ nhớ.
%     \item $(\alpha\beta)_{ij}$ cho biết liệu ảnh hưởng của loại GPU có phụ thuộc vào hãng hay không.
% \end{itemize}

% Nếu tương tác tồn tại, điều đó có nghĩa là:
% \textit{tác động của loại GPU không giống nhau giữa các hãng.}

% \textbf{Phân rã tổng biến thiên}

% Tổng biến thiên trong dữ liệu được đo bằng Tổng bình phương sai số (Total Sum of Squares):
% \[
%     SS_{\text{Total}} = \sum_{i}\sum_{j}\sum_{k} (Y_{ijk} - \bar{Y}_{...})^2
% \]

% ANOVA tách tổng biến thiên này thành 4 phần:
% \[
%     SS_{\text{Total}} = SS_{\text{Manufacturer}}
%     + SS_{\text{Dedicated}}
%     + SS_{\text{Interaction}}
%     + SS_{\text{Error}}
% \]

% \paragraph{1. Biến thiên do hãng}
% \[
%     SS_{\text{Manufacturer}}
%     = \sum_{i} n_i (\bar{Y}_{i..} - \bar{Y}_{...})^2
% \]
% Đại diện cho mức độ khác biệt giữa các hãng.

% \paragraph{2. Biến thiên do loại GPU}
% \[
%     SS_{\text{Dedicated}}
%     = \sum_{j} n_j (\bar{Y}_{.j.} - \bar{Y}_{...})^2
% \]
% Đại diện cho mức độ khác biệt giữa GPU chuyên dụng và GPU tích hợp.

% \paragraph{3. Biến thiên do tương tác}
% \[
%     SS_{\text{Interaction}} =
%     \sum_{i}\sum_{j}
%     n_{ij}(\bar{Y}_{ij.} - \bar{Y}_{i..} - \bar{Y}_{.j.} + \bar{Y}_{...})^2
% \]
% Mục đích: kiểm tra xem hai yếu tố có \textit{phụ thuộc lẫn nhau} hay không.

% \paragraph{4. Biến thiên ngẫu nhiên}
% \[
%     SS_{\text{Error}} =
%     \sum_{i}\sum_{j}\sum_{k}
%     (Y_{ijk} - \bar{Y}_{ij.})^2
% \]

% \textbf{Tính toán F-statistic}

% Đầu tiên tính Mean Square (MS):
% \[
%     MS = \frac{SS}{df}
% \]

% Sau đó thống kê F:
% \[
%     F_{\text{Manufacturer}} =
%     \frac{MS_{\text{Manufacturer}}}{MS_{\text{Error}}}
% \]

% \[
%     F_{\text{Dedicated}} =
%     \frac{MS_{\text{Dedicated}}}{MS_{\text{Error}}}
% \]

% \[
%     F_{\text{Interaction}} =
%     \frac{MS_{\text{Interaction}}}{MS_{\text{Error}}}
% \]

% \textbf{Ý nghĩa thống kê}
% Nếu một giá trị F lớn, nghĩa là:
% \[
%     \text{Biến thiên do yếu tố } > \text{biến thiên ngẫu nhiên}
% \]
% → yếu tố đó có tác động thật sự lên băng thông bộ nhớ.

% Ngược lại, nếu F nhỏ → p-value lớn → yếu tố không ảnh hưởng đáng kể.

% ==========================================================
\subsubsection{Quy trình thực hiện ANOVA hai yếu tố}

\textbf{Bước 1: Nhập dữ liệu}
Chuyển biến Manufacturer(Hãng) và Dedicated(Loại GPU) thành biến phân loại trong R:
\begin{lstlisting}[language=R, caption={Chuyển biến phân loại trong R}, captionpos=b]
    main_df$Manufacturer <- as.factor(main_df$Manufacturer)
    main_df$Dedicated    <- as.factor(main_df$Dedicated)
\end{lstlisting}

\textbf{Bước 2: Kiểm tra cấu trúc dữ liệu}

Đảm bảo Manufacturer và Dedicated là biến phân loại, sử dụng lệnh:
\begin{lstlisting}[language=R, caption={Kiểm tra cấu trúc dữ liệu}, captionpos=b]
    str(main_df)
\end{lstlisting}

\begin{figure}[H]
    \begin{center}
        \includegraphics[width=10cm]{Images/factor_check.png}
    \end{center}
    \caption{Kết quả kiểm tra kiểu dữ liệu với lệnh str() trong R}
    \label{fig:strdata}
\end{figure}

Để kiểm tra các thông số cơ bản của các biến cần kiểm định, sử dụng lệnh:

\begin{lstlisting}[language=R, caption={Kiểm tra thông số cơ bản của các biến}, captionpos=b]
summary(main_df$Memory_Bandwidth)
table(main_df$Manufacturer)
table(main_df$Dedicated)
\end{lstlisting}
Kết quả:
\begin{figure}[H]
    \begin{center}
        \includegraphics[width=0.4\textwidth]{Images/basicstats.png}
    \end{center}
    \caption{Thống kê mô tả cơ bản của biến Memory\_Bandwidth(Băng thông bộ nhớ)}
    \label{fig:basicstats}
\end{figure}

Tóm tắt thống kê cho biến \textbf{Memory\_Bandwidth}:
\[
    \min = 2.0,\ \ Q1 = 29.9,\ \ \mathrm{Median} = 112.0,\ \ \mathrm{Mean} = 144.1,\ \ Q3 = 219.4,\ \ \max = 1280.
\]

Nhận xét:
\begin{itemize}
    \item Băng thông có phân bố lệch phải mạnh (max = 1280).
    \item Số mẫu theo hãng rất lệch (AMD 1179, ATI 78, Intel 153, Nvidia 1618).
    \item GPU chuyên dụng (Yes) chiếm 2812/3028 mẫu, lệch phân bố so với GPU tích hợp (No = 216).
\end{itemize}

\textbf{Bước 3: Thực hiện thống kê ANOVA hai yếu tố}
\[
    \text{Memory\_Bandwidth} \sim \text{Manufacturer} * \text{Dedicated}
\]

Xây dựng mô hình ANOVA hai yếu tố trong R:

\begin{lstlisting}[language=R, caption={Xây dựng mô hình ANOVA hai yếu tố trong R}, captionpos=b]
    anova_model <- aov(Memory_Bandwidth ~ Manufacturer * Dedicated, data = main_df)
\end{lstlisting}

\textbf{Kiểm tra giả định:}

\begin{itemize}
    \item [a)] Kiểm định chuẩn hoá phần dư bằng Shapiro-Wilk test:
          \begin{table}[H]
              \centering
              \begin{tabular}{lll}
                  Giả thuyết không $H_0$ & : & Phần dư có phân phối chuẩn.       \\
                  Giả thuyết đối $H_1$   & : & Phần dư không có phân phối chuẩn.
              \end{tabular}
          \end{table}
          \vspace{-10pt}
          Ta sử dụng hàm \verb|shapiro.test| để kiểm tra:
          \begin{lstlisting}[language=R, caption={Kiểm định chuẩn hoá phần dư bằng Shapiro-Wilk test}, captionpos=b]
    shapiro.test(residuals(anova_model))
          \end{lstlisting}
          \begin{figure}[H]
              \begin{center}
                  \includegraphics[width=10cm]{Images/shapiro_test.png}
              \end{center}
              \label{fig:shapiro_test}
          \end{figure}
          \textbf{Nhận xét:} Vì $p-value < 0.05$ nên ta bác bỏ $H_0$, thừa nhận $H_1$, do đó phần dư không tuân theo phân phối chuẩn.
    \item [b)] Kiểm định phương sai đồng nhất bằng Levene’s test:
          \begin{table}[H]
              \centering
              \begin{tabular}{lll}
                  Giả thuyết không $H_0$ & : & Phương sai của các nhóm là bằng nhau (đồng nhất). \\
                  Giả thuyết đối $H_1$   & : & Có ít nhất hai nhóm có phương sai khác nhau.
              \end{tabular}
          \end{table}
          \vspace{-10pt}
          Ta sử dụng hàm \verb|leveneTest| để kiểm tra:
          \begin{lstlisting}[language=R, caption={Kiểm định phương sai đồng nhất bằng Levene’s test}, captionpos=b]
    library(car)
    leveneTest(Memory_Bandwidth ~ Manufacturer * Dedicated, data = main_df)
          \end{lstlisting}
          \begin{figure}[H]
              \begin{center}
                  \includegraphics[width=14cm]{Images/levene_test.png}
              \end{center}
              \caption{Kết quả kiểm định Levene về tính đồng nhất phương sai}
              \label{fig:levene_test}
          \end{figure}

          \textbf{Nhận xét:}Vì $p-value < 0.05$ nên ta bác bỏ $H_0$, thừa nhận $H_1$, do đó phương sai giữa các nhóm không đồng nhất.
\end{itemize}
\textbf{Kết luận về giả định:}
\begin{itemize}
    \item Cả hai giả định quan trọng của ANOVA đều bị vi phạm.
    \item Tuy ANOVA vẫn thường “mạnh” với kích thước mẫu lớn (n = 3028),
          nhưng kết luận cần diễn giải thận trọng.
\end{itemize}


\newpage
Bảng ANOVA từ R:
\begin{figure}[H]
    \begin{center}
        \includegraphics[width=14cm]{Images/anovatable.png}
    \end{center}
    \caption{Bảng ANOVA hai yếu tố được tạo bởi mô hình trong R}
    \label{fig:anovatable}
\end{figure}

Ta có thể vẽ lại bảng như sau:

\begin{table}[H]
    \centering
    \caption{Bảng phân tích phương sai hai yếu tố đối với biến \textit{Memory\_Bandwidth}}
    \begin{tabular}{lccccc}
        \toprule
        \textbf{Nguồn biến thiên} & \textbf{SS}           & \textbf{df} & \textbf{MS} & \textbf{F} & \textbf{p-value} \\
        \midrule
        Manufacturer
                                  & 2{,}798{,}908
                                  & 3
                                  & 932{,}969
                                  & 54.088
                                  & $< 2\times 10^{-16}$                                                              \\
        Dedicated
                                  & 879{,}716
                                  & 1
                                  & 879{,}716
                                  & 51.001
                                  & $1.15\times 10^{-12}$                                                             \\
        Tương tác (Manufacturer × Dedicated)
                                  & 83{,}757
                                  & 2
                                  & 41{,}878
                                  & 2.428
                                  & 0.0884                                                                            \\
        Sai số
                                  & 52{,}109{,}247
                                  & 3021
                                  & 17{,}249
                                  & --
                                  & --                                                                                \\
        \bottomrule
    \end{tabular}
\end{table}


\textbf{Nhận xét:}
\begin{itemize}
    \item \textbf{Manufacturer} có ảnh hưởng rất đáng kể đến băng thông bộ nhớ.
    \item \textbf{Dedicated} có ảnh hưởng rất đáng kể.
    \item \textbf{Tương tác Manufacturer × Dedicated không có ý nghĩa thống kê} (p = 0.0884 > 0.05).
\end{itemize}
\newpage

\textbf{Bước 4: Kiểm định giả định}

\paragraph*{(a) Kiểm định chuẩn hoá phần dư}

Từ kết quả kiểm định Shapiro--Wilk mà đoạn code trên tạo ra, ta thấy rằng:
\[
    W = 0.85823,\quad p < 2.2\times 10^{-16}
\]

Và vẽ được biểu đồ Q-Q bằng cách sử dụng lệnh:
\begin{verbatim}
qqnorm(residuals(anova_model))
qqline(residuals(anova_model), col = "red")
\end{verbatim}

Kết quả như sau:
\begin{figure}[H]
    \begin{center}
        \includegraphics[width=14cm]{Images/qqplot.png}
    \end{center}
    \caption{Biểu đồ Q-Q kiểm tra phân phối chuẩn của phần dư}
    \label{fig:qqplot}
\end{figure}



\textbf{Nhận xét:} Phần dư \textbf{không} tuân theo phân phối chuẩn.
(Biểu đồ Q-Q cũng cho thấy lệch phải mạnh.)

\paragraph*{(b) Kiểm định phương sai đồng nhất (Levene)}

Từ kết quả kiểm định Levene mà đoạn code trên tạo ra, ta thấy rằng:
\[
    F = 34.63,\quad p < 2.2\times 10^{-16}
\]

\textbf{Nhận xét:} Phương sai giữa các nhóm \textbf{không đồng nhất}.



\textbf{Bước 5: Kiểm định hậu nghiệm}
Sử dụng lệnh:
\begin{verbatim}
TukeyHSD(anova_model)
\end{verbatim}

\paragraph*{(1) Hãng sản xuất}

\begin{figure}[H]
    \begin{center}
        \includegraphics[width=10cm]{Images/manufacturer.png}
    \end{center}
    \caption{Kết quả kiểm định Tukey HSD cho yếu tố Hãng sản xuất(Manufacturer)}
    \label{fig:tukey_manufacturer}
\end{figure}


Từ kết quả trên, ta thấy rằng tất cả các cặp hãng đều khác biệt có ý nghĩa thống kê, ví dụ:
\[
    \text{Intel - AMD}: -110.80,\ p < 10^{-7}
\]
\[
    \text{ATI - AMD}: +74.35,\ p = 8\cdot 10^{-6}
\]
\[
    \text{Nvidia - AMD}: +19.78,\ p = 0.00049
\]

\textbf{Nhận xét:}
Băng thông bộ nhớ trung bình khác biệt đáng kể giữa các hãng.


\paragraph*{(2) Loại GPU}

\begin{figure}[H]
    \begin{center}
        \includegraphics[width=10cm]{Images/dedicated.png}
    \end{center}
    \caption{Kết quả kiểm định Tukey HSD cho yếu tố loại GPU (Dedicated)}
    \label{fig:tukey_dedicated}
\end{figure}

Từ kết quả trên, ta thấy rằng:
\[
    \text{Yes - No} = 38.94,\quad p = 2.76 \times 10^{-5}
\]

\textbf{Nhận xét:} GPU chuyên dụng có băng thông cao hơn GPU tích hợp.

\newpage
\paragraph*{(3) Tương tác}

\begin{figure}[H]
    \begin{center}
        \includegraphics[width=14cm]{Images/interaction.png}
    \end{center}
    \caption{Kết quả kiểm định Tukey HSD cho tương tác Hãng $\times$ Loại GPU}
    \label{fig:tukey_interaction}
\end{figure}

Từ kết quả trên, ta thấy rằng hầu hết các cặp đều không có ý nghĩa thống kê (nhiều trường hợp NA do thiếu dữ liệu).

\textbf{Nhận xét:} Không tìm thấy bằng chứng thuyết phục rằng tác động giữa hãng và loại GPU ảnh hưởng băng thông theo cách khác biệt đáng kể.

\newpage

\textbf{Bước 6: Vẽ đồ thị}
Biểu đồ tương tác giữa Hãng và Loại GPU:
\begin{figure}[H]
    \begin{center}
        \includegraphics[width=14cm]{Images/interaction_plot.png}
    \end{center}
    \caption{Biểu đồ tương tác giữa Hãng và Loại GPU}
    \label{fig:interaction_plot}
\end{figure}

\newpage
Biểu đồ trực quan hóa băng thông bộ nhớ theo Hãng và Loại GPU:
\begin{figure}[H]
    \begin{center}
        \includegraphics[width=14cm]{Images/anova_memory_bandwidth_plot.png}
    \end{center}
    \caption{Biểu đồ trực quan hóa băng thông bộ nhớ theo Hãng và Loại GPU}
    \label{fig:anova_mb_plot}
\end{figure}

\newpage
\subsection{Kết luận tổng hợp}
Dựa trên kết quả phân tích:
\begin{itemize}
    \item \textbf{Hãng sản xuất có ảnh hưởng rất mạnh} đến băng thông bộ nhớ (p < 2e-16).
    \item \textbf{Loại GPU (Chuyên dụng/Tích hợp) cũng có ảnh hưởng đáng kể} (p = 1.15e-12).
    \item \textbf{Không có tương tác rõ rệt} giữa hai yếu tố (p = 0.0884).
    \item Tuy nhiên, \textbf{giả định chuẩn hoá phần dư và đồng nhất phương sai đều bị vi phạm}.
          Điều này không vô hiệu hoá ANOVA do mẫu rất lớn, nhưng kết luận nên được diễn giải cẩn trọng.
    \item Phân tích hậu nghiệm cho thấy tất cả hãng đều khác nhau có ý nghĩa, và GPU chuyên dụng vượt trội GPU tích hợp.
\end{itemize}

\textbf{Tóm tắt:}
\textit{Hãng sản xuất và loại GPU đều ảnh hưởng mạnh đến băng thông bộ nhớ, nhưng không có tương tác giữa hai yếu tố. Tuy nhiên, phân bố phần dư cho thấy ANOVA có thể bị ảnh hưởng bởi vi phạm giả định, do đó có thể xem xét thêm biến đổi dữ liệu (log-transform) trong phân tích mở rộng.}
\newpage
\section{Thảo luận và mở rộng}

\subsection{Thảo luận}

Bài báo cáo đã cung cấp cơ sở dữ liệu và phân tích các thuộc tính về GPU cho ta thấy cái nhìn bao quát và hiểu rõ hơn các yếu tố ảnh hưởng đến hiệu suất của GPU. Các mô hình được xây dựng để dự đoán có thể đánh giá hiệu năng GPU thông qua các thông số kỹ thuật. Sau khi phân tích, ta có một số nhận xét sau:


\begin{itemize}
    \item Thống kê mô tả:
          \begin{itemize}
              \item Cho thấy sự phân bố không đồng đều của các thông số, đặc biệt là  Memory\_Bandwidth-thông số quyết định hiệu suất xử lý của GPU phần lớn giá trị tập trung ở mức thấp.
              \item Các biến số như Memory\_Speed, Memory\_Bus, L2\_Cache có ảnh hưởng nhất định đến Memory\_Bandwidth, tuy nhiên mức độ ảnh hưởng khác nhau và có sự phân tán lớn trong dữ liệu.
              \item Trong dữ liệu chỉ gồm có một số ít nhà sản xuất chiếm đa số, điều này có thể ảnh hưởng đến tính tổng quát của mô hình dự đoán.
              \item Dữ liệu có một số ngoại lai, điều này có thể ảnh hưởng đến trực tiếp đến kết quả phân tích và các mô hình dự đoán.
          \end{itemize}

    \item Thống kê suy diễn:

          \begin{itemize}
              \item Mô hình hồi quy tuyến tính đa biến cho thấy các biến Process, Memory, Memory\_Speed, Memory\_Bus, L2\_Cache có ảnh hưởng đáng kể đến Memory\_Bandwidth, tuy nhiên mô hình có R-squared không cao, chỉ khoảng 0.8419, do đó đây chưa phải là mô hình hồi quy tốt nhất.
              \item Như đã đề cập ở trên, phần lớn các biến xét đến trong các mô hình và phân tích đều không tuân theo phân phối chuẩn, điều này ảnh hưởng đến tính chính xác của các mô hình dự đoán, buộc nhóm phải tìm các giải pháp để cải thiện các mô hình đó.
          \end{itemize}
\end{itemize}


\subsection{Mở rộng}

\textbf{Cải thiện mô hình hồi quy tuyến tính bội.}

Như đã đề cập ở phần thống kê suy diễn, mô hình hồi quy tuyến tính bội hiện tại có R-squared không cao, chỉ khoảng 0.8419 vì biến Memory\_Bandwidth không tuân theo phân phối chuẩn. Do đó Ta có thể thử dùng phép biết đổi logarit ($\log$) cho các biến để ổn định phương sai của các sai số, làm cho độ phân tán của sai số giảm xuống, phương sai trở nên đồng nhất hơn. Vì trong dữ liệu có giá trị 0 ở một số biến (VD: L2\_Cache) nên ta sẽ lấy $\log(X+1)$ để tránh giá trị lỗi. Ta thực hiện như sau:

\begin{lstlisting}[language=R, caption=Phép biến đổi logarit cho các biến.]
training_set_log <- training_set
testing_set_log <- testing_set
training_set_log$Memory <- log(training_set_log$Memory + 1)
training_set_log$Memory_Bus <- log(training_set_log$Memory_Bus + 1)
training_set_log$Memory_Bandwidth <- log(training_set_log$Memory_Bandwidth +
                                           1)
training_set_log$Memory_Speed <- log(training_set_log$Memory_Speed + 1)
training_set_log$L2_Cache <- log(training_set_log$L2_Cache + 1)
training_set_log$Process <- log(training_set_log$Process + 1)
testing_set_log$Memory <- log(testing_set_log$Memory + 1)
testing_set_log$Memory_Bus <- log(testing_set_log$Memory_Bus + 1)
testing_set_log$Memory_Bandwidth <- log(testing_set_log$Memory_Bandwidth +
                                          1)
testing_set_log$Memory_Speed <- log(testing_set_log$Memory_Speed + 1)
testing_set_log$L2_Cache <- log(testing_set_log$L2_Cache + 1)
testing_set_log$Process <- log(testing_set_log$Process + 1)
new_model <- lm(
  Memory_Bandwidth ~ Process + Memory + Memory_Speed
  + Memory_Bus + L2_Cache,
  training_set_log
)
summary(new_model)
\end{lstlisting}

\underline{\textbf{Kết quả:}}

\begin{figure}[H]
    \centering
    \includegraphics[width=0.8\textwidth]{graphics/KQ2.png}
    \caption{Kết quả mô hình hồi quy tuyến tính bội sau khi biến đổi logarit}
    \label{fig:6_1model}
\end{figure}

\underline{\textbf{Kết luận:}}

Mặc dù mô hình có thật sự cải thiện hơn so với trước với R-squared là 0.8874, tuy nhiên p\_value của biến Process lớn hơn 0.05, nên ta sẽ loại bỏ biến này trong mô hình mới.

Xây dựng lại mô hình mới được kết quả như sau:

\begin{figure}[H]
    \centering
    \includegraphics[width=0.6\textwidth]{graphics/KQ3.png}
    \caption{Kết quả mô hình hồi quy tuyến tính bội sau khi loại bỏ biến Process}
    \label{fig:6_2model}
\end{figure}

\underline{\textbf{Nhận xét:}} có thể dễ dàng nhận thấy, tất cả các chỉ số đã có sự cải thiện rõ rệt so với các mô hình trước. Cụ thể R-squared đã tăng lên 0.9067. Do đó, mô hình hồi quy hiện tại hoạt động hiệu quả, chính xác và đáng tin cậy hơn.

\textbf{Dự đoán giá trị Memory\_Bandwidth sử dụng mô hình mới.}

\begin{lstlisting}[language=R, caption=Dự đoán giá trị Memory\_Bandwidth sử dụng mô hình mới.]
pred_log_values <- predict(new_model, newdata = testing_set_log)

# Chuyen tu log ve gia tri binh thuong
pred_original_scale <- exp(pred_log_values) - 1

# Lay gia tri thuc te
actual_original_scale <- testing_set$Memory_Bandwidth

# Tao df cho de nhin
results_summary_display_original <- data.frame(
  Memory_Bandwidth_Actual = actual_original_scale,
  Memory_Bandwidth_Predict = pred_original_scale,
  Error_Original = actual_original_scale - pred_original_scale # Tinh sai so
)

head(results_summary_display_original, 10)
\end{lstlisting}

\underline{\textbf{Kết quả:}}

\begin{figure}[H]
    \centering
    \includegraphics[width=0.8\textwidth]{graphics/Test.png}
    \caption{Kết quả dự đoán giá trị Memory\_Bandwidth sử dụng mô hình mới}
    \label{fig:6_2predict}
\end{figure}

\begin{figure}[H]
    \centering
    \includegraphics[width=0.8\textwidth]{graphics/6_1better.png}
    \caption{Đồ thị giữa giá trị thực tế và dự đoán của Memory\_Bandwidth (mô hình cải tiến)}
    \label{fig:6_3error}
\end{figure}

\begin{figure}[H]
    \centering
    \includegraphics[width=0.8\textwidth]{graphics/rmse_better.png}
    \caption{MRSE và RR của mô hình cải tiến}
    \label{fig:mrse_rr_better}
\end{figure}

\underline{\textbf{Nhận xét:}}

Có thể thấy rằng mô hình sau khi biến đổi logarit đã cải thiện đáng kể độ chính xác trong việc dự đoán giá trị Memory\_Bandwidth. Đồ thị ở hình \ref{fig:6_3error} cho thấy các điểm dữ liệu tập trung gần đường chéo $y=x$ hơn so với mô hình gốc ở hình \ref{fig:kq2_hqtt}. Mặc dù vậy khi tính ngược lại giá trị gốc (bằng cách lấy hàm mũ), ta thấy rằng MRSE của mô hình mới là 41.0669 cao hơn so với mô hình gốc (33.2738), mặc dù vậy RR của mô hình mới lại cao hơn so với mô hình gốc.

\section{Nguồn dữ liệu và nguồn code}
\begin{itemize}
  \item \textbf{Nguồn dữ liệu:} Dữ liệu được lấy từ trang Kaggle tại địa chỉ: \href{https://www.kaggle.com/datasets/iliassekkaf/computerparts/data}{Kaggle}.
  \item \textbf{Nguồn code:} Toàn bộ mã nguồn R được sử dụng trong báo cáo này đều được viết bởi nhóm sinh viên và tham khảo từ các tài liệu, bài báo, và nguồn học thuật khác để thực hiện báo cáo.
\end{itemize}

\clearpage                   % Sang trang mới
\phantomsection
\begin{thebibliography}{99}  % Số 99 nghĩa là danh sách tối đa 99 mục
  % Tài liệu [2]
  \bibitem{nguyendinhhuy2019}
  Nguyễn Đình Huy, \textit{Giáo trình Xác suất và Thống kê}, NXB Đại học Quốc gia TP.HCM, 2019.

  % Tài liệu [4]
  \bibitem{nguyenkieudung_slide}
  Nguyễn Kiều Dung, \textit{Bài giảng Xác suất Thống kê}, Tài liệu lưu hành nội bộ, Trường Đại học Bách Khoa - ĐHQG TP.HCM, 2024.

\end{thebibliography}

\end{document}